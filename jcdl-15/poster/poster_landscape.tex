\documentclass[landscape,final,a0paper,fontscale=0.285]{baposter}

\usepackage{calc}
\usepackage{graphicx}
\usepackage{amsmath}
\usepackage{amssymb}
\usepackage{relsize}
\usepackage{multirow}
\usepackage{rotating}
\usepackage{bm}
\usepackage{url}

\usepackage{graphicx}
\usepackage{multicol}

%\usepackage{times}
%\usepackage{helvet}
%\usepackage{bookman}
\usepackage{palatino}

\newcommand{\captionfont}{\footnotesize}

\graphicspath{{images/}{../images/}}
\usetikzlibrary{calc}

\newcommand{\SET}[1]  {\ensuremath{\mathcal{#1}}}
\newcommand{\MAT}[1]  {\ensuremath{\boldsymbol{#1}}}
\newcommand{\VEC}[1]  {\ensuremath{\boldsymbol{#1}}}
\newcommand{\Video}{\SET{V}}
\newcommand{\video}{\VEC{f}}
\newcommand{\track}{x}
\newcommand{\Track}{\SET T}
\newcommand{\LMs}{\SET L}
\newcommand{\lm}{l}
\newcommand{\PosE}{\SET P}
\newcommand{\posE}{\VEC p}
\newcommand{\negE}{\VEC n}
\newcommand{\NegE}{\SET N}
\newcommand{\Occluded}{\SET O}
\newcommand{\occluded}{o}

%%%%%%%%%%%%%%%%%%%%%%%%%%%%%%%%%%%%%%%%%%%%%%%%%%%%%%%%%%%%%%%%%%%%%%%%%%%%%%%%
% Multicol Settings
%%%%%%%%%%%%%%%%%%%%%%%%%%%%%%%%%%%%%%%%%%%%%%%%%%%%%%%%%%%%%%%%%%%%%%%%%%%%%%%%
\setlength{\columnsep}{1.5em}
\setlength{\columnseprule}{0mm}

%%%%%%%%%%%%%%%%%%%%%%%%%%%%%%%%%%%%%%%%%%%%%%%%%%%%%%%%%%%%%%%%%%%%%%%%%%%%%%
%%% Begin of Document
%%%%%%%%%%%%%%%%%%%%%%%%%%%%%%%%%%%%%%%%%%%%%%%%%%%%%%%%%%%%%%%%%%%%%%%%%%%%%%

\begin{document}

%%%%%%%%%%%%%%%%%%%%%%%%%%%%%%%%%%%%%%%%%%%%%%%%%%%%%%%%%%%%%%%%%%%%%%%%%%%%%%
%%% Here starts the poster
%%%---------------------------------------------------------------------------
%%% Format it to your taste with the options
%%%%%%%%%%%%%%%%%%%%%%%%%%%%%%%%%%%%%%%%%%%%%%%%%%%%%%%%%%%%%%%%%%%%%%%%%%%%%%
% Define some colors

%\definecolor{lightblue}{cmyk}{0.83,0.24,0,0.12}
%\definecolor{lightblue}{rgb}{0.145,0.6666,1}
\definecolor{dbrary}{RGB}{134, 190, 179}

%%
\begin{poster}%
  % Poster Options
  {
  % Show grid to help with alignment
  grid=false,
  % Column spacing
  colspacing=1em,
  columns=3,
  % Color style
  bgColorOne=white,
  bgColorTwo=white,
  borderColor=dbrary,
  headerColorOne=dbrary,
  headerColorTwo=dbrary,
  headerFontColor=white,
  boxColorOne=white,
  boxColorTwo=dbrary,
  % Format of textbox
  textborder=roundedleft,
  % Format of text header
  eyecatcher=true,
  headerborder=closed,
  headerheight=0.1\textheight,
%  textfont=\sc, An example of changing the text font
  headershape=roundedright,
  headershade=shadelr,
  headerfont=\Large\bf\textsc, %Sans Serif
  textfont={\setlength{\parindent}{1.5em}},
  boxshade=plain,
%  background=shade-tb,
  background=plain,
  linewidth=2pt
  }
  % Eye Catcher
  {
    \includegraphics[height=5em]{img/datavyu-leaf-large.png}
  } 
  %Title
  {\bf{Databrary: Enabling Sharing and Reuse of Research Video} }
  % Authors
  { Andrew S. Gordon \emph{(drew@databrary.org)}, Rick O. Gilmore, Karen E. Adolph, David S. Millman, Lisa Steiger, and Dylan A. Simon \\
  JCDL 2015}
  % University logo
  {
    \includegraphics[height=5em]{img/databrary-leaf-large.png}
  }

%%%%%%%%%%%%%%%%%%%%%%%%%%%%%%%%%%%%%%%%%%%%%%%%%%%%%%%%%%%%%%%%%%%%%%%%%%%%%%
%%% Now define the boxes that make up the poster
%%%---------------------------------------------------------------------------
%%% Each box has a name and can be placed absolutely or relatively.
%%% The only inconvenience is that you can only specify a relative position 
%%% towards an already declared box. So if you have a box attached to the 
%%% bottom, one to the top and a third one which should be in between, you 
%%% have to specify the top and bottom boxes before you specify the middle 
%%% box.
%%%%%%%%%%%%%%%%%%%%%%%%%%%%%%%%%%%%%%%%%%%%%%%%%%%%%%%%%%%%%%%%%%%%%%%%%%%%%%

%%%%%%%%%%%%%%%%%%%%%%%%%%%%%%%%%%%%%%%%%%%%%%%%%%%%%%%%%%%%%%%%%%%%%%%%%%%%%%
  \headerbox{Abstract}{name=abstract,column=0,row=0}{
%%%%%%%%%%%%%%%%%%%%%%%%%%%%%%%%%%%%%%%%%%%%%%%%%%%%%%%%%%%%%%%%%%%%%%%%%%%%%%
      Video and audio recordings serve as a primary data source in many fields, especially in the social and behavioral sciences.
      Recordings present unique opportunities for reuse and reanalysis for novel scientific purposes, but also present challenges related to respecting the privacy of individuals depicted.
      Databrary is a web-based service for sharing and reusing the video data created by researchers in the developmental and learning sciences.
      By investigating how researchers organize, analyze, and mine their own recordings, we have implemented a system that empowers researchers to capture, store, and share recordings in a standardized way.
      The Databrary service promotes storage, management, sharing, and reuse of research data, controls access privileges to restricted human subject data, and facilitates browsing and discoverability of datasets.
 }

%%%%%%%%%%%%%%%%%%%%%%%%%%%%%%%%%%%%%%%%%%%%%%%%%%%%%%%%%%%%%%%%%%%%%%%%%%%%%%
  \headerbox{Access Restrictions}{name=access,column=0,row=1, below=abstract}{
%%%%%%%%%%%%%%%%%%%%%%%%%%%%%%%%%%%%%%%%%%%%%%%%%%%%%%%%%%%%%%%%%%%%%%%%%%%%%%
      Databrary ensures that videos with identifiable information can be shared under four different release levels:
        \begin{center}
        \includegraphics[width=\textwidth]{img/release_levels.png}
        \end{center}
  }

%%%%%%%%%%%%%%%%%%%%%%%%%%%%%%%%%%%%%%%%%%%%%%%%%%%%%%%%%%%%%%%%%%%%%%%%%%%%%%
  \headerbox{What Can Be Shared}{name=shared,column=0,below=access}{
%%%%%%%%%%%%%%%%%%%%%%%%%%%%%%%%%%%%%%%%%%%%%%%%%%%%%%%%%%%%%%%%%%%%%%%%%%%%%%
    Databrary can house video, audio, PDF, spreadsheet, image, and text-based files along with associated metadata and annotations.
    Video files are transcoded into standard, HTML5-compatible formats, currently H.264+AAC in MP4.
  }


%%%%%%%%%%%%%%%%%%%%%%%%%%%%%%%%%%%%%%%%%%%%%%%%%%%%%%%%%%%%%%%%%%%%%%%%%%%%%%
\headerbox{Databrary.org}{name=splash,column=1,span=1,row=0}{
  %%%%%%%%%%%%%%%%%%%%%%%%%%%%%%%%%%%%%%%%%%%%%%%%%%%%%%%%%%%%%%%%%%%%%%%%%%%%%%
  \begin{center}
    \includegraphics[width=\textwidth]{img/databrary-splash.png}
  \end{center}
  % Databrary's splash page shows a timeline of recent activity, a set of clickable tags that link to studies with those search terms, a featured dataset, and links that enable users to search for data or manage their own data.
}

%%%%%%%%%%%%%%%%%%%%%%%%%%%%%%%%%%%%%%%%%%%%%%%%%%%%%%%%%%%%%%%%%%%%%%%%%%%%%%
  \headerbox{Technology Stack}{name=stack,column=2,span=1,row=0,bottomaligned=splash}{
%%%%%%%%%%%%%%%%%%%%%%%%%%%%%%%%%%%%%%%%%%%%%%%%%%%%%%%%%%%%%%%%%%%%%%%%%%%%%%
    Databrary is built on PostreSQL using the Scala Play framework and AngularJS.
    Data are preserved indefinitely in a secure data storage facility at NYU. 
    There is no cost to use the system; a sustainability model is under development.
    All code is GPL on GitHub.
    \begin{center}
    \includegraphics[width=.9\textwidth]{img/architecture.png}
    \end{center}
  }


%%%%%%%%%%%%%%%%%%%%%%%%%%%%%%%%%%%%%%%%%%%%%%%%%%%%%%%%%%%%%%%%%%%%%%%%%%%%%%
  \headerbox{Acknowledgments}{name=thanks,column=0,span=2,above=bottom}{
%%%%%%%%%%%%%%%%%%%%%%%%%%%%%%%%%%%%%%%%%%%%%%%%%%%%%%%%%%%%%%%%%%%%%%%%%%%%%%
  Databrary is based on work supported by the National Science Foundation under Grant No. BCS-1238599 and the Eunice Kennedy Shriver National Institute of Child Health and Human Development under Cooperative Agreement U01-HD-076595. 
  Any opinions, findings, and conclusions or recommendations expressed in the material contributed here are those of the author(s) and do not necessarily reflect the views of the National Science Foundation or the Eunice Kennedy Shriver National Institute of Child Health and Human Development.
  }

%%%%%%%%%%%%%%%%%%%%%%%%%%%%%%%%%%%%%%%%%%%%%%%%%%%%%%%%%%%%%%%%%%%%%%%%%%%%%%
\headerbox{Session-level Data Management}{name=data,column=1,span=1,below=splash,above=thanks}{
  %%%%%%%%%%%%%%%%%%%%%%%%%%%%%%%%%%%%%%%%%%%%%%%%%%%%%%%%%%%%%%%%%%%%%%%%%%%%%%
  \begin{center}
    \includegraphics[width=.9\textwidth]{img/spreadsheet.png}
  \end{center}
  % Users may upload video and related data \emph{as they collect it}, recording information about individual data collection sessions such as participant characteristics, sharing permission levels, and study conditions.
}

%%%%%%%%%%%%%%%%%%%%%%%%%%%%%%%%%%%%%%%%%%%%%%%%%%%%%%%%%%%%%%%%%%%%%%%%%%%%%%
\headerbox{Timeline}{name=data,column=2,span=1,row=1,below=stack, above=bottom, aligned=data}{
  %%%%%%%%%%%%%%%%%%%%%%%%%%%%%%%%%%%%%%%%%%%%%%%%%%%%%%%%%%%%%%%%%%%%%%%%%%%%%%
  \begin{center}
  \includegraphics[width=.95\textwidth]{img/timeline.png}
  \end{center}
  % Within a session, data are represented on a timeline that reflects when events occurred. This is particularly useful for studies with multiple camera views (e.g., eye-tracking) or with combined behavioral and image/display streams.
}

\end{poster}

\end{document}
