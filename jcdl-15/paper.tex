\documentclass{sig-alternate}

\begin{document}
\conferenceinfo{JCDL}{'15 Knoxville, Tennesse USA}

\title{Databrary: A digital data library for sharing research video}

\numberofauthors{3}

\author{
\alignauthor
Dylan A. Simon\\
	\affaddr{Databrary Project}\\
	\affaddr{New York University}\\
	\affaddr{196 Mercer Street, Room 807}\\
	\affaddr{New York, NY 10012}\\
	\email{dylan@databrary.org}
\alignauthor
Drew Gordon\\
	\affaddr{Databrary Project}\\
	\affaddr{New York University}
	\affaddr{196 Mercer Street, Room 807}\\
	\affaddr{New York, NY 10012}\\
	\email{drew@databrary.org}
\alignauthor
Rick O. Gilmore, Ph.D.\\
	\affaddr{The Pennsylvania State University}\\
	\affaddr{Department of Psychology}\\
	\affaddr{University Park, PA 16802}\\
	\email{rogilmore@psu.edu}
}

\maketitle

\begin{abstract}
%%dg being that the conference them is big data, should we be trying to be more explicit about that?
% big data is generally referring to "more rows" rather than larger files. I think the closest we get is the size of video files
% which is already noted below. Just wondering if it makes sense to be more explicit or not.
Video captures the complexity, richness, and diversity of behavior
unlike any other tool. As a result, large numbers of researchers who
study human or animal behavior employ video. Video documents itself to a
large degree, and therefore has significant potential for data re-use.
Still, researchers rarely share video. Video often contains information
about personal identities, so considerations about research ethics pose
challenges to sharing. The relatively large size of video files and
diversity of formats pose technical challenges. In this paper, we will
describe how the web-based Databrary data library has overcome the most
significant barriers to sharing video within the developmental and
learning sciences community, including solutions to maintaining
participant privacy, data tagging, and data management.
%%dg also preservation and discoverability - though maybe that's included in data management
\end{abstract}

\category{H.2.8}{Database Management}{Database Applications}[image
databases, scientific databases]

\terms{Management, Documentation, Design, Security, Legal Aspects}

\keywords{Data sharing, open science, video, psychology, developmental science}

\section{Introduction}

Video plays a key role in many areas of human research, particularly in developmental psychology, education, and is even a primary data source ...
%%dg unclear what is meant by primary data source in this context
Unfortunately, in standard practice, the use of these video data is isolated to the researchers who collected them because of the perceived requirements about privacy restrictions.
The goal of the Databrary project is to establish policies and tools to allow video data to be documented, organized, and shared so that other researchers can discover and re-purpose existing videos for their own research interest.
Accordingly, Databrary, with a focus on developmental psychology research, has created a web-based library to accommodate the storage and organization of video data, capture contextual information necessary for re-use, and standardize across datasets without placing undue burden on data contributors.

% this is questionable...
Most existing research data repositories provide either unstructured storage of arbitrary files, or have rigorously structured schema that require contributors to properly extract and format contributions.
%%dg should pick examples to illustrate both sides of the spectrum (ICPSR - ???)
While the former is well suited for archival of entire datasets and general-purpose hosting of materials, it does not facilitate discoverability, understanding, or re-use across various datasets.
%%dg i'd say that the general purpose ones are fine on discoverability, but they might no facilitate re-use or reflect very much of the original 
%context of the data
The latter, on the other hand, generate large-scale, homogonous data sources for meta- and re-analysis, but limited to specific domains and research questions, losing the full richness and innovation of the source materials and field.
Databrary attempts to strike a balance by creating a common structure that is general enough to provide the flexibility for incorporating heterogenous datasets while at the same time facilitating a depth of access to configure data more to the original context in which it was collected and analyzed. While the architecture of the backend allows us to curate and include diverse datasets, another important component is a well-define user experience that allows researchers to define how their data is kept in Databrary by uploading it as they collect it. 
The following provides further detail on the design and decision-making of this structure.
% we're a hybrid?
%%dg more like a system informed by the pitfalls of committing solely to either extreme.


%%dg perhaps better to frame below as the affordances and challenges of working with video data. 
%That is..."A key component to understanding the design decisions made in creating Databrary is the unique affordances and challenges of video and audio data"
%Affordances
We have intentionally chosen to limit the bulk of the data hosted in the library to human-consumable, self-documenting, timeseries data: video and audio.
These data can and largely are consumed and analyzed, at least initially, without sophisticated analysis tools: once digitally decoded, humans can glean the bulk of the richness of such data directly.
Similarly, these data can often be used on their own or with minimal explanation, as the nature of the activity and participants are readily apparent.
This naturally aids in the problems of documentation and re-use that plague most flat-file or dense datasets.
%Challenges
On the other hand, since automated analysis of videos is an active yet immature research area, the problems of search, discoverability, and meta-analysis over these media remain.
We attempt to address these issues by augmenting the primary video data with additional, structured annotations and metadata.
%%dg probably need to make this section more cohesive, but the main gist is 1) We focus on video data, 2)video data has these inherent challenges. 3) At the same time there are already challenges involved in creating a research data repository, independant of the format of the data travels on. 4) This paper dicusses the challenges we have resolved and the challenges we have coming up.

\section{Data Schema}

%%dg would a figure illustration be helpful here? do we already have one?

As with most user-driven databases, we separate individual data contributions or packages into individual collections variously referred to as datasets, studies, or volumes.
Volumes can be associated with a minimal amount of metadata (title, description, permissions, etc.) as well as additional files (documents and other materials).
These volumes are to some extent embeddable and referenceable, both among themselves and to and from external entities by DOI.
The remainder of this work will address the available structures of data within these volumes, highlighting how these structures allow commonalities between volumes to be represented.

% should we actually use the term "record"?  Alternatives: "grouping", "label", "annotation", ...

There are two key, innovative structural elements that comprise a volume: the session and the record.
Sessions are the primary container for data files, basically an elaboration on traditional file folders with additional temporal metadata.
Records are used to represent all types of scalar variable-based data, traditionally contained in flat files or databases, as well as providing an organizational structure among sessions.
The combination of these two elements allows data and metadata to be flexibly related and annotated.

\subsection{The ``Session''}

(Behvioral|Developmental) Science Researchers record activities with individuals either for the purpose of a later, closer review or analysis, or simply to document and verify measurements and observations made during the collection.
Thus, the video itself is often the core item of research, representing a single data collection, experiment, trial, visit, or case depending on the nature of the research.
However, in many cases a single session can comprise multiple videos: from serial data streams due to recording equipment being reset or producing multiple files, and/or in parallel from multiple data capture sources that overlap in time.
These primary recordings may also be augmented with other forms of timeseries data such as motion capture, eye tracking, or physiological recordings, all of which have specific and critical temporal relationships with the primary data.

% should we be using more language like, "we observed", "in practice"?
When dealing with multiple primary recordings, researchers often reduce these raw files into a single representation.
For example, they often concatenate serial recordings, or make a composite video by joining or overlaying frames.
Secondary data is also often overlaid or combined in this way, such as by adding an additional waveform or frequency visualization below a video, or by overlaying crosshairs representing location or gaze data.
Doing so allows for more convenient human analysis of the data in standard tools that only support a single data stream, but also often loses some of the raw data, by obscuring, masking, or most often downsampling spatially or temporally.

Instead, we capture these relationships within an object called a session, which can comprise not only arbitrary files but also their absolute or relative temporal positioning.
The metaphor we use here is a timeline, wherein each session defines a time range, with absolute timestamps representing the exact time of data collection, or a date of collection and relative positioning defined by a 0 starting point of the collection (or potentially any other meaningful wall-time units).
Files placed within a session can then be (optionally) positioned, much as in video or audio editing software.
Although some advanced recording equipment provides automatic timestamping or synchronization than can be used, most researchers collecting multiple parallel data streams often use ``sync points'' such as an electronic trigger signal, flash, or tone to align recordings.

Once the timeframe of a session is established, this can allow temporally-positioned annotations to be attached to the session.
% this is sort of an aside:
%%dg maybe the value here is that we've designed things with this specific type of research in mind.
In fact, this practice is very common in research when analyzing recordings, and specialized desktop software tools are employed to review video and audio and annotate events, utterances, actions, or phases at any temporal scale, from individual samples/frames to the entire session.
Most of these tools currently focus on a single recording file, however, and despite the plethora of tools and features, there is little interoperability.

In addition to providing an organizational structure that allows users to visualize data in ways familiar to their research, we also currently allow keyword-based tagging and comment-based discussions on any temporal segment of a session.
These annotations could be applied manually by users or by automated video analysis systems.
We expect that these annotations will provide an index target for additional discovery capabilities for users looking for particular features within data.

%%dg, i guess it's crucial to make sure the audience understands that one of the specific challenges for us is that we're not just storing comprehensive single video files, but multiple files that capture the same phenomenon and how do we keep all those together in a meaningful way...to differentiate us from being "just another you tube" .. which i guess is what is already being said above. Though might need to do a little more hand holding for the reader as to why this is an issue in general and why it's important to even those who might be faced with different types of video data to collect (? maybe I'm rambling now ?).


\subsection{The ``Record''}

An object called a record serves a number of purposes around session and metadata organization.
A record itself represents any number of variables of measurements---what would traditionally be a row of a database or flat file.
For example, a record may represent an individual participant within the dataset, and contain their subject identifier, gender, birthdate, and any other demographic, survey, or measurement data about that participant.
A record could also represent a particular experimental task, condition, location, or outcome, and contain variables with corresponding parameters or descriptions.
Records are thus labeled and divided by the type of entity they represent.
These variables themselves a simply textual, numeric, date, or potentially any other scalar data as in any traditional tabular data interface, with the records of a given type constituting the rows.

However, records can also be used to organize sessions.
We observed that different researchers organize their data files in different ways according to the type of research.
Many consider participants to be the primary item of analysis, and so create a directory for each participant.
Others organize their data based on date, experimental condition, phase of analysis, or age groups.
Sometimes even single video files are split up into segments according to phase or activity, and researchers consider this the primary property organizational principal.

Rather than trying to handle all of these strategies separately, we instead leverage these records.
That is, records can be attached to sessions, or, equivalently, sessions can be grouped into records, in a many-to-many relationship.
For example, in a traditional psychology study, each participant may appear in exactly one session, so the collected information about that participant, added to the record, is attached one-to-one with their session.
A dataset may be composed of a few different sub-sets based on experimental procedures or location, and here each session would be placed into exactly one of these few records.
In some research, multiple participants may also appear in multiple sessions, such as recordings of the same class of students over a span of time.

Since researchers can create whichever record types apply to their data, they can then choose to group their data by whichever record type they wish.
This also allows users to either import or export data in what they consider the native directory structure by indicating the appropriate nesting of record types.
By providing some standardization of record types, we can also perform broad, meta-searches across the library, for example for all videos containing participants of a particular age range.

Finally, records can be attached not only to entire sessions, but also to arbitrary temporal sub-sets of a session.
That is, if any record only applies to part of a session, for example a single activity, or a participant being present for only part of a recording, it can be applied within the timeline of a session to only the appropriate part.
This allows the final type of organizational structure, wherein records define virtual sub-sessions that can be grouped and interacted with as units, without needing to cut up source videos.
It also can be used to represent annotations at a finer temporal scale, for example individual events interactions that are normally handled by off-line annotation softaware.
However, the appropriate interfaces and tools to input or import these annotations are still being developed.

\section{Standardization vs. Flexibility}

As the extant research data libraries demonstrate, there is always the tension between standardization and flexibility.
Since we're primarily dealing with video and audio data, we have, without much loss of flexibility, standardized on a particular encoding format (codecs and parameters) that allows for convenient access and HTML5-based streaming, while still allowing arbitrary resolutions and sample/frame-rates.
We also allow for ``dumps'' using certain portable data formats to include any data that cannot be encompased by our standard formats, reasoning that these data may be relevant for re-use, but not broadly adopted by the community.

Here we have also introduced a number of metadata concepts where this tension becomes most apparent, namely the available record types, variables (and values within categorical data types), and tag keywords.
We have adopted an approach of encouraging adherence to existing standardized options but also allowing new types to be created as the data require it.
We encourage standardization both on-line as data are collected and entered via our user interface using auto-completion, suggestions, and drop-down options, and off-line via manual review and curaion of contributed data.

Achieving the appropriate balance will no doubt be challanging, but by limiting our focus to particular research domains we have been able to achieve a good amount of standardization across the entire data library without overly burdening contributors.



\section{Introduction}

Film and video recording have been a mainstay of research in the
developmental sciences for nearly 100 years [CITE]. A typical lab
acquires, on average, 12 hours of video per week depicting behavior in
natural or laboratory environments [CITE]. These recordings constitute a
primary source of raw data. Prior to descriptive or statistical
analyses, researchers score or “code” recordings, mostly manually, using
commercial software or makeshift tools. However, even the most detailed
and laborious video coding does not exhaust the possibilities for
analyses by other researchers with different questions. Thus, sharing
video promises a wide range use cases, perhaps more than sharing
flat-file data or static images. Despite the wealth of untapped
information available in research videos and the labor expended in
collecting and coding them, developmental researchers rarely share data.
In the prevailing culture, investigators conduct research privately
within their own laboratories. Data are destroyed or molder away on a
shelf after the research is published. The Databrary project aims to
change the status quo, making sharing standard practice.

Databrary enables researchers to share video and associated flat-file
data for research and educational purposes. For example, a researcher
interested in language development might reuse a set of video recordings
created originally to study the development of walking. Researchers can
conduct integrative analyses by combining data across disparate studies,
increase sample sizes, verify the feasibility of proposed projects,
compare methods, learn about procedures, and independently verify
results. Similarly, investigators can reuse shared videos for
educational purposes such as to find illustrative examples for teaching,
use exemplars in presentations, show students how to conduct procedures,
or provide supplemental materials for reviewers.

Data sharing and open science practices are emerging or accepted norms
in the biological, physical and earth sciences with demonstrated
benefits for scientific transparency and accelerated discovery \cite{Curry2011,iedadata,Kaye2009,Overpeck2011}.
Open research practices are gaining traction in the behavioral and
social sciences \cite{Lunshof2014,Adolph2012,King2011,APS2012,Nosek2012i,Nosek2012ii}. However, open data sharing among developmental
researchers is still the exception \cite{MacWhinney2001} not common practice. In seeking
to expand sharing of video-based data, Databrary faces challenges
concerning the interplay among institutional research policies, ethical
issues related to obtaining participants’ consent for sharing, and
technical authorization systems. This paper describes the current state
of the Databrary project including the project organization, how content
is acquired, stored, preserved, and accessed, policy considerations, and
how the library has affected and hopes to affect the behavioral science
research community.

\section{Project Organization}

Databrary is a joint project of New York University (NYU) and The
Pennsylvania State University (Penn State). The Databrary project began
with a workshop on open video data sharing funded by the National
Science Foundation (NSF) under Grant No. BCS-1139702 at which
developmental scientists, computer scientists, library scientists, and
federal agency program officers discussed the promises and challenges of
sharing video data. NSF and the National Institute of Child Health and
Human Development (NICHD) have provided funding under grant No.
BCS-1238599 and Cooperative Agreement U01-HD-076595, respectively. The
interdisciplinary team bring expertise in developmental and neural
science, information technology, library science, software development,
data curation, and community engagement. In addition, a board of expert
advisors composed of developmental scientists, library scientists, and
leaders of data repositories in the behavioral and social sciences
provide guidance.

The project is organized around five core activities.

\subsection{Curation}

To seed the Databrary, staff identified and selected video datasets for
manual curation. The datasets were selected based on availability,
presumed interest to developmental science community, suitability for
the organization and file structure of the repository, diversity of
study type, and on the availability of appropriate sharing permissions
(see 2.2 below). Manual curation of archival datasets continues under
the direction of a staff member with expertise in library and
information sciences.

Nevertheless, we have found manual data curation time consuming and
difficult, especially for researchers. The more difficult and
time-consuming data sharing becomes, the less likely it will be that
researchers participate despite best intentions. So, we have created
user-friendly spreadsheet-like interfaces that enable researchers to
self-curate datasets as they are collected. At the end of a data
collection session, researchers may upload videos and other data files
collected along with metadata about participant demographic
characteristics or experimental conditions. Databrary transcodes videos
to a standard format in the background, and the system alerts the user
when the video can be previewed. Researchers may also upload study-level
materials -- code books, flat data files used for statistical analyses,
images of experimental equipment, links to related external information
(e.g., GitHub repositories), etc.

This scheme, which we call "upload-as-you-go", makes the data curation
required for sharing part of the normal scientific workflow, not a
burdensome after thought. No data are shared until the researcher grants
access, but unlike the *ad hoc* organization in a typical investigator's
laboratory, on Databrary, the data are organized in a standardized way
that could be readily and easily shared with others when the
investigator is ready to do so.

Sharing video requires special considerations because the data cannot be
de-identified without reducing its value. No previously established
research data repository has specialized in video data sharing.
Therefore, Databrary had to develop a new policy and permission
framework for sharing identifiable and sensitive data while
simultaneously preserving the rights of research participants.

The framework has two components, both of which build on
well-established principles. One extends the idea of informed consent to
participate in research by requiring that participants give explicit
permission for their identifiable or sensitive data to be shared with
other researchers. Databrary makes available on its website template
language researchers may use for seeking permission to share, along with
video demonstrations and additional guidance. So, researchers who wish
to share video must either use the template or demonstrate that
participants have granted an equivalent level of permission. The second
component involves limiting access to individuals who have prior formal
authorization. Researchers must either secure authorization to access
Databrary from their institution, typically a university where they
conduct independent research, or must have another authorized researcher
grant them access. Institution-level authorization requires a formal
agreement with Databrary. Fundamentally, the Databrary framework ensures
that participants control whether other researchers can view their video
(or other identifiable data), and only researchers who pledge to uphold
ethical principles, including maintaining participant confidentiality,
may access Databrary's sensitive materials.

Databrary developed these policies and agreements in close consultation
with Institutional Review Boards (IRBs), sponsored research privacy
administrators, and legal staff at NYU and Penn State, with other
experts in research data ethics, and with Federal officials. See
http://databrary.org/access/guide.html for more details.

The centerpiece of the Databrary project is the repository itself. Our
team of programmers has developed digital library software, including
asset management, workflow, access control, discovery services, and a
user interface. This work is closely coordinated with the central NYU IT
infrastructure providers for services such as high-capacity storage,
high-security storage, and high-performance computing (for video
processing). The long-term goal is to build the cyber-infrastructure for
seamless search, streaming, uploading, and downloading and to provide an
open-source framework for other entities to build on the Databrary
software or extend it to other content domains.

\subsection{Annotation and Metadata}

Video becomes most useful for conducting research when it is transformed
into quantitative and qualitative data in a flat-file format. Thus, the
Databrary team is enhancing a general-purpose video-coding software
application specialized for exploring, annotating, tagging, scoring, and
visualizing the video files in preparation for qualitative and
quantitative analyses. Toward that end, we have released Datavyu
(datayvu.org), a free, open source, multi-platform desktop application.
Datavyu is a fork of the OpenSHAPA tool used by some developmental
researchers. The long-term goal is to build a web-based version of
Datavyu that will provide both data analysis and data management
services required for preparing and organizing video material prior to
its storage in Databrary.

\subsection{Community Practices}

The current zeitgeist in the developmental science community is one of
privacy and data hoarding. A primary activity of the Databrary project
is to introduce the practice of open data sharing to the developmental
science community. We are doing so through announcements to
developmental research societies, colloquia, and conference
presentations. The team participates in workshops to discuss sharing
issues and hosts events to teach potential users about sharing and
analysis tools. We post links on websites frequented by developmental
researchers and participate in user forums. In addition, we have asked
leaders in developmental science to set an example for other researchers
by publicly committing to open video data sharing. More than a hundred
have responded to the call
(http://databrary.org/community/contributors.html). Our long-term goal
is a self-sustaining community of researchers committed to open
science.

\section{Building the Library}

Existing data management practices pose challenges for video sharing.
Despite similar research methods, study designs vary widely. No two
developmental science labs manage data in the same way. Some studies are
longitudinal, where researchers observe the same participants at
multiple sessions. Some are cross-sectional, where researchers observe
each participant at only one session. The timing of observations may be
determined by participants’ age (e.g., 4-month-olds, 8-month-olds,
12-month-olds) or abilities (e.g., preverbal, one-word utterances,
sentences), experimentally determined variables (e.g., pre- and
post-intervention), or other factors. Some labs organize longitudinal
data by grouping files first by participant and then by session date.
Other labs organize longitudinal data by session and then by
participant. Still others organize longitudinal data based on task (book
reading, block building, free play). Some researchers institute a
central data management system to be followed by all the lab members,
providing easier access and greater transparency for the entire lab but
not necessarily providing the structure for similar benefits to
researchers unfamiliar with the lab’s practices. Other researchers allow
their students to keep separate records, making it difficult to share
data even within the lab. Some researchers keep videos, metadata, and
analyses together, and some do not. Idiosyncratic terms, record-keeping,
and data management systems are the norm. Databrary must enable
researchers to discover and understand each other’s materials,
regardless of the original investigator’s data management system.

Databrary takes several approaches to address this challenge: A flexible
data model, informal metadata contributions by users via tagging and
commenting, mediated curation, and incentives to deposit materials more
consistently. Figure 1 illustrates the principal data model. It
accommodates different hierarchies and organizations, such as those
described in the examples above.

The full expressive flexibility of the data model is not evident to
Databrary users. Instead, users see different aspects of the data model
at different times, depending on the context. Researchers organize their
materials by acquisition date and time into structures called sessions.
A session corresponds to a unique recording episode and contains one or
more recordings or related flat files (assets). Researchers can group
sessions in different ways, using whatever groupings are appropriate for
the particular dataset. These groupings may also identify particular
time segments of a recording to distinguish tasks, events, or
participants that comprise the session. In the data model, the aggregate
of these flexible groupings is called a volume. In practice, researchers
combine sessions or segments of sessions within and across datasets to
form the raw material that are subsequently described in published
articles and presentations. Thus, Databrary contributors can combine
sessions or segments within and across datasets with ancillary
materials, such as coding manuals, Datavyu spreadsheets, statistical
analyses, questionnaires, IRB documents, computer code, sample displays,
and links to published journal articles in an aggregate structure
investigators call a study. Like datasets, studies are represented as
volumes in the data model.

The discovery and browsing interfaces offer content items at the volume
(study or dataset) level to users searching the library. The
contributor-assigned groupings (records), which may include relevant
metadata about participants (measures, such as participant demographic
information, domain-specific survey information, location data,
condition variables, task descriptions, or other properties), allow a
second level of filtering and organization within and between relevant
identified studies or datasets. This enables users to quickly identify
data that may be relevant to their own interests or research.

Authorized Databrary Investigators can also add keyword tags at the
study/dataset, session, or segment level, and can endorse or deprecate
keywords that have been suggested by other authorized researchers. These
abilities are critical because different researchers may have very
different reasons for citing a study or using data. Future
implementations may extend the ability of authorized researchers to
annotate studies with other kinds of metadata.

The Databrary team recognizes that other data repositories enforce
strict metadata ontologies and that doing so may have benefits in
sub-domains of research when there is community consensus \cite{cogatlas}. We will
support standard data coding ontologies among researchers, but only
enforce standardization for a small set of standard tags such as study
date, participant birth date, and sex. We will also encourage
contributors to report race, ethnicity, primary language, language of
dataset, and location of session. The Databrary team chose not to
require strict metadata ontologies for several reasons. We hope to
reduce the pre-deposit curational demands on contributors, encourage the
repurposing of shared data, and foster the rapid adoption of data
sharing. Developing and achieving agreement on metadata ontologies can
take significant amounts of time. Video data are so rich and complex
that in many domains, researchers have not settled on standard
definitions for particular behaviors and may have little current need
for standard tasks, procedures, or terminology. However, we will also
encourage users to re-use tags and terminology by suggesting common or
similar terms, without confining users to these suggestions. User
communities within Databrary may eventually converge on common
conceptual and metadata ontologies based on the most common (and
commonly endorsed) keyword tags, but standardized ontologies are not
necessary for browsing and searching in most of the use cases we
envision.

Our experiences curating and ingesting archival datasets have
highlighted the considerable value of contributors entering raw video
data into Databrary as soon as recordings are acquired. Immediate
uploading reduces the workload on investigators, minimizes the risk of
data loss and corruption, and accelerates the speed with which materials
become openly available. To encourage immediate uploading, Databrary
provides a complete set of controls so that researchers can restrict
access to their data to only their own labs or to other users of their
choosing. Datasets can be shared at a later point when data collection
and ancillary materials are complete, whenever the contributor is
comfortable sharing, or when journals or funders require it. Databrary
has published a Data Sharing Manifesto \cite{manifesto} that explains to researchers
the Databrary philosophy. Standards about when data should be shared are
evolving. Our philosophy is consistent with concepts and practices in
other domains where data sharing is the norm (e.g., www.iedata.org).
Planned enhancements to the Datavyu tool will allow contributors to
organize video files and metadata as part of the analysis process. This
will facilitate the contribution of packaged datasets to Databrary.

As part of the curation process, Databrary stores at least two versions
of each item of Databrary video content: a copy for access, and the
received original file if it was digital, or a 10-bit YUV digital
preservation copy if the original version was not digital. Currently,
the access version format is H.264 (HiP) with AAC audio in an MPEG-4
container, although we expect the appropriate video formats to change
over time, as has been the case with many digital video formats in
recent years.

For preservation, the original file (if digital) or the preservation
copy will be stored in a long-term preservation repository managed
jointly by the NYU Libraries and the central Information Technology
Services unit. This repository ensures that each content item has a METS
\cite{METS} structural metadata file that associates the digital asset with its
metadata. It stores files in two mirrored and geographically distributed
locations, and a third copy on offsite tape; it performs regular fixity
checks; and it provides a format migration capacity, in the event that a
stored format becomes at risk of obsolescence.

\section{Policies Framework}

Sharing video recordings poses a unique challenge to existing data
sharing policies because videos contain personally identifying
information—specifically participants’ faces and voices and often the
insides of their homes and classrooms. Sharing personally identifiable
information puts research participants and others depicted in recordings
at increased risk for loss of privacy. At the same time, blurring or
altering original recordings to hide identities undermines or eliminates
their value to other researchers. Often, participants’ faces and voices
produce the behaviors of interest. So, Databrary has elected to maximize
the potential for data re-use by keeping recordings in their original
unaltered form. Instead of removing participants’ identities, Databrary
restricts access to identifiable or sensitive data to authorized
researchers. Further, Databrary provides access only when the people
depicted have given permission for their information to be shared with
other researchers.

\subsection{Restricting Access}

Databrary provides access to shared data only to authorized researchers
who have agreed to uphold common practices concerning the responsible
and ethical use of identifiable and sensitive data. To become an
authorized investigator, applicants must register on the site and
electronically sign the Authorized Investigator Agreement, which must
also be co-signed by the applicant’s institution. Full privileges will
be granted only to researchers with principal investigator (PI) status
at their institutions. Other researchers may be granted privileges if
they are affiliated with a PI who agrees to sponsor and supervise their
application. Initially there will be a manual process to identify the
institutional representative—typically the authorizing official of the
university—who can co-sign the Investigator Agreement. However, as the
user groups at each university expand, Databrary may implement
administrative accounts at each institution. This will enable the
authorizing official to independently manage the authorizations of
individual researchers at her institution.

\subsection{Seeking permission to share identifiable data}

Data from a particular session may be stored in Databrary for the
contributing researcher’s use whether the records are shared with other
scientists or not. When a researcher chooses to share, Databrary makes
records openly available to the community of authorized researchers only
if the people depicted in the recordings have given permission to
release the data for sharing. Thus, Databrary requires that people
depicted in recordings grant permission before their information can be
shared. Databrary’s policies extend currently accepted principles of
informed consent to the situation where participants are granted
authority to consent to (or refuse) the release of their identifiable
data.

We developed these ideas in close collaboration with the NYU and PSU IRB
staff. To formalize the process of acquiring permission, we developed a
Participant Release Form Template, based on photo or video release
language many researchers use currently. The template release form has
standard language that Databrary recommends investigators should use
with study participants. This language makes it easy for participants to
understand what is involved in sharing their video data, with whom it
will be shared, and the potential risks associated with releasing their
video and other identifiable data to other researchers. Use of the
template also allows for the standardization of language associated with
the release of identifiable or sensitive participant data.

\subsection{Technical assistance with IRBs}

Some IRBs may deem an investigator’s existing, approved video or photo
release form equivalent to the Databrary release. This enables a
researcher to share with Databrary recordings they have already
collected. However, most researchers will need to modify their research
protocols, by adding the Databrary sharing permission procedures, prior
to collecting new shareable video data. Databrary staff are available to
advise potential data contributors about how to amend existing research
protocols so that the information acquired is Databrary-compliant.
Protocol amendments involve seeking approval for use of the Databrary
template release form and modifying the time period over which collected
data will be made available. Specifically, researchers must remove any
clauses in research consent documents that require data destruction
after some fixed period of time since Databrary intends to store shared
data indefinitely.

\section{Impact}

The Databrary project has already begun to have an impact. As of
mid-January 2015, more than 57 scientists from 35 institutions in North
America, the UK, and Europe have received authorization for full access
to and data sharing with Databrary. More than 50 additional researchers
are in the process of securing institutional approval. In consultation
with Databrary staff, many of these researchers have secured or are in
the process of securing permission from their research ethics boards to
share archival or new data. Thus, the Databrary project has laid the
groundwork for change in scientific culture around data sharing within
the community of developmental science researchers who are most familiar
with it. In a similar vein, Databrary is attempting to lead by example
where open science practices are concerned. Databrary is an open source
project. The entire code base is available on Github
(github.com/databrary) as are all policy documents. The project team
consults regularly with other leaders in the open science, data sharing,
and data repository communities, and we share best practices among them.

Databrary has also begun to have an impact in the policy arena. The
notion that identifiable research data may be shared – under the right
circumstances – is not new to Databrary \cite{HCP,PGP}. But, Databrary has
created a set of policies and template documents that will help IRBs to
come to see that seeking permission to share data is merely an extension
of the principle of informed consent. Furthermore, Databrary’s
Authorized Investigator Agreement combines provisions for data
contribution with those guiding data use, part of an overall effort to
reduce barriers to sharing and re-use. Our colleagues in the data
sharing community tell us that this combination represents an innovation
in itself.

The Databrary team has come to understand that laboratory data
management practices pose significant challenges to widespread data
sharing. Simply put, many researchers deploy workflows that would
require significant modification in order for video files and associated
metadata to be readily shared. In response, Databrary has expanded its
data curation expertise and capacity. However, hand-curating significant
volumes of research data in this way is not sustainable in the long
term. Building tools that can enable self-curation will be essential to
the future success and sustainability of Databrary. As a result, the
team has begun to compile best practices for data management and
curation that will be folded into future data coding (Datavyu) and data
library (Databrary) features. We note that the process of understanding
the diversity of data management techniques in developmental science
poses unique opportunities for librarians and curators to learn how
researchers organize and manage their data for daily use. It also
affords researchers opportunities to learn from their library science
colleagues about the value of adopting best practices in data
organization and management.

\section{Conclusions}

The Databrary project aims to increase scientific transparency and
accelerate discovery in developmental science by creating a
user-friendly and powerful infrastructure for researchers to share video
and related data. Clearly, sharing video data poses technical and policy
challenges, but it presents significant opportunities for accelerating
discovery if these challenges can be met successfully. The Databrary
project has already made significant strides in identifying and
overcoming many of the obstacles, and the tools and infrastructure that
we develop promise to enhance data sharing and management practices in
the entire behavioral science community.

\section{Acknowledgments}

This work was supported by the National Science Foundation (BCS-1238599)
and the National Institute of Child Health and Human Development
(U01-HD-076595-01). The authors gratefully acknowledge the NYU
Libraries, Human Connectome Project, the Personal Genome Project, the
Inter-university Consortium for Political and Social Research, the
Center for Open Science, Dataverse, Data Dryad, and staff from the
National Database for Autism Research for their valuable advice and
consultation.

\bibliographystyle{abbrv}
\bibliography{references}

\end{document}
