% rog We need to cut a half page or more...

\documentclass{sig-alternate}

\begin{document}
\conferenceinfo{JCDL}{'15 Knoxville, Tennessee USA}

\title{Databrary: A digital data library for sharing research video}

\numberofauthors{3}

\author{
\alignauthor
Dylan A. Simon\\
	\affaddr{Databrary Project}\\
	\affaddr{New York University}\\
	\affaddr{New York, NY, USA}\\
	\email{dylan@databrary.org}
\alignauthor
Drew Gordon\\
	\affaddr{Databrary Project}\\
	\affaddr{New York University}\\
	\affaddr{New York, NY, USA}\\
	\email{drew@databrary.org}
\alignauthor
Rick O. Gilmore\\
	\affaddr{Penn State}\\
	\affaddr{Department of Psychology}\\
	\affaddr{University Park, PA, USA}\\
	\email{rogilmore@psu.edu}
}

\maketitle

\begin{abstract}
Video and audio recordings serve as a primary data source in many fields, especially in the social and behavioral sciences.
Recordings present unique opportunities for reuse and reanalysis for novel scientific purposes.
However, there are numerous technical and policy challenges preventing existing recordings from being accessed and reused by other researchers.
By investigating how researchers organize, analyze, and mine their own recordings, we have implemented a system that empowers researchers to capture, store, and share these data in a standardized way.
There are remaining challenges around discovery and annotation, and we discuss some options and initial approaches.
While many of Databrary's strengths have derived from its focus on a particular scholarly domain, many of these issues have broader relevance to the digital library community.
\end{abstract}

\category{H.2.8}{Database Management}{Database Applications}[scientific databases, image databases]
\category{H.3.5}{Information Storage and Retrieval}{Online Information Systems}[data sharing, web-based services]

\terms{Design, Standardization}

\keywords{Data sharing, open science, video, psychology, developmental science}

\section{Introduction}

Video and audio recordings serve as a primary data source in psychology, linguistics, education, anthropology, and numerous other fields, especially the study of human and animal development.
Unfortunately, researchers who collect video rarely share recordings with others, due largely to perceived requirements associated with privacy restrictions.
The goal of the Databrary project is to increase the repurposing and reuse of recordings, especially video, by creating tools that allow video to be documented, organized, and shared among researchers.
We have created a web-based data library to accommodate the storage and organization of video data, capture contextual information necessary for reuse, and standardize across datasets without placing undue burden on data contributors.
In this paper, we discuss the design and technical challenges of the system and describe some findings and structural solutions to these challenges.

\subsection{Balancing Generality and Specificity}

Research data repositories require a balance between generality and specificity.
Archives that permit the unstructured storage of arbitrary files lie at one extreme;
those that present a rigorously structured schema and required contribution formats lie at the other.
The flexibility of general-purpose repositories make them especially well-suited to the archival of entire datasets, but these systems do not facilitate search, understanding, or re-use without extensive, usually labor-intensive, curation \cite{Peer_2012}.
Targeted repositories, on the other hand, generate large-scale, homogeneous data sources that facilitate meta- and re-analysis, but serve only specific domains. 
We have attempted to strike a balance between these extremes by making video and audio recordings the unifying focus of Databrary. This allows additional structured and unstructured resources to be built around and discovered through this functional core.

\subsection{Video-driven research}

Video and audio recordings constitute a form of human-consumable, self-documenting, timeseries data.
Recordings can be and largely are consumed and analyzed, at least initially, without the use of sophisticated tools: once digitally decoded, trained human observers can directly glean much of the data's richness.
Video data collected in one context can often be used by others for different purposes with minimal explanation.
This naturally reduces the problems of documentation and re-use that plague many flat-file datasets or repositories for data that require proprietary software to view.
However, since automated analysis of videos is an active yet still immature research area, the problems of search, discoverability, and meta-analysis over these media remain \cite{Albertson_2013}. In short, recordings pose challenges that older, less dynamic media do not \cite{Lanagan_Smeaton_2012}.

Research using recordings typically proceeds with the addition of temporally-positioned annotations, either by direct observation or with the aid of simple visualization or analysis tools.
Specialized desktop software tools --- both commercial (e.g., Noldus Observer, Mangold Interact, and Studiocode) and academic (e.g. Transana, Elan, CLAN, and Datavyu) --- enable researchers to annotate events, measurements, utterances, or behaviors at any temporal scale, from individual samples/frames to the entire session.
Most of these tools currently focus on a single recording, and despite the plethora of tools and features, there is little interoperability.
Moreover, with rare exceptions \cite{MacWhinney2001} there are few standards for annotations themselves, even across studies performed by the same researcher. %Adolph, personal communication?

\subsection{Approach}

Our approach has been to provide consistent structure related to the storage and presentation of primary recordings while providing contributors the flexibility that they have come to expect from their own, possibly \emph{ad hoc}, archival systems.
To keep datasets intact, Databrary associates heterogeneous flat-file data, materials, and references with the primary video or audio recordings.
We provide a standard structure for recordings and their associated metadata, including context, contents, and annotations. This facilitates the storage and retrieval of data more closely mapped to the original context in which it was collected and analyzed.
We achieve this conformance for previously collected datasets through a combination of in-depth investigative curation and automated, server-side ingest processes.
For datasets that are in the process of being collected, a well-defined user experience provides a platform for researchers to actively curate their own research data as they acquire it.
The following section provides specific details about the design and decision-making behind this unifying structure.

\section{Data Architecture}

At the top level of organization, we separate individual data contributions into collections or packages called \emph{volumes}.
Most researchers in the target domain organize their research activity around the collection and analysis of specific datasets; scholarly products like papers or presentations draw upon one or more datasets. 
The volume enables Databrary to reflect the many-to-many structural relationship between datasets and studies in a flexible and powerful way.
Volumes can be associated with a minimal amount of metadata (title, description, permissions, etc.) as well as additional files (documents and other materials).
These volumes are citeable in that they are persistently identified (with a URI) and can cite external resources using a Digital Object Identifier (DOI) and an Open Researcher and Contributor ID (ORCID).

Two structural elements comprise a volume: the \emph{session} and the \emph{record}.
Sessions provide the primary container for data files, an elaboration on traditional file folders with additional temporal metadata. 
Records represent all types of scalar variable-based data, traditionally contained in flat files or databases, as well as provide an organizational structure among sessions.
The combination of these two elements allows data and metadata to be flexibly related and annotated.

\subsection{Sessions}

The session draws upon documented practices in the ways researchers use video. 
It was designed to organize recordings and related data streams in a way familiar to researchers, keeping data in its rawest and most flexible form without requiring  modifications that reduce the precision or utility of the data to others.

Recordings are often the core component of data collection.
Recordings provide raw materials for subsequent detailed analysis or documentation about what measurements were made during the collection \cite{Bakeman_2012}.
In some cases a single session can comprise multiple videos, either serially in time representing different segments of activity or in parallel from multiple cameras recording the same events.
These primary recordings may also be augmented with other forms of timeseries data such as motion capture, eye tracking, or physiological recordings, as well as with additional point-in-time measurements, observations, or surveys, all of which have specific and critical temporal relationships with the primary data.

When working with multiple primary recordings, researchers often combine these raw files into a single representation.
For example, they may concatenate serial recordings or make a composite video by joining or overlaying frames.
Some may overlay or combine secondary data, by adding waveform or frequency visualizations below a video, or by overlaying cross-hairs representing location or gaze data.
Doing so allows for more convenient human analysis of the data in standard playback tools that support single data streams, but this spatial or temporal downsampling can obscure or mask some data patterns.

A session container captures the relationships among data elements from a single data acquisition, capturing heterogeneous file types \emph{and} their absolute or relative temporal positioning.
We use a timeline metaphor for storage and visualization: Each session defines a time range, with absolute timestamps representing the exact time of data collection or relative positioning with an optional date.
Files placed within a session can then be (optionally) positioned, much as in video or audio editing software.
Although some advanced recording equipment provides automatic timestamping or synchronization features, most researchers who collect parallel data streams use ``sync points'' such as an electronic trigger signal, flash, or tone to align recordings.
These sync points can then be used to manually place recordings on the timeline.
In this way, the session container represents the structure of a data collection session but retains significant flexibility.

\subsection{Records}

Record objects represent any number of variables, measurements, or items of analysis within a session: What would traditionally be a row of a database or flat file.
A record may represent an individual participant and contain a subject identifier, gender, birth date, and demographic, survey, or measurement data about that person.
It could also represent a particular experimental trial, task, condition, location, or outcome, and contain variables with corresponding measurements, parameters, or descriptions.
Accordingly, records are labeled and organized by the type of entity they represent.
The variables themselves are textual, numeric, date, or any scalar data form.

Records help to organize sessions.
We have observed that researchers organize data files in substantially different ways.
Many consider participants to be the primary focus of analysis, and so create a filesystem directory for each participant.
Others organize their data based on date, experimental condition, phase of analysis, or age group.
Sometimes even single video files are split up into segments according to scene or activity, and researchers consider this property the primary organizational principle.

Rather than trying to handle all of these strategies separately, Databrary leverages the power of records.
Records can be attached to sessions, or, sessions can be grouped into records, in a many-to-many relationship.
For example, in one type of lab-based psychology study, a participant may appear in exactly one session, so the collected information about that participant, added to a record, is attached in a one-to-one relationship with a specific session.
In other research, multiple participants may appear in multiple sessions, such as recordings of the same class of students over a span of time.
A dataset may consist of distinct sub-sets of participants based on differences in experimental procedures or locations; here each session would be placed into a record representing the procedure or location.

Since researchers can create whichever record types apply to their data, they can also choose to group their data as they wish.
This allows users to import data in what they consider the native directory structure by indicating the appropriate nesting of record types along with record variables themselves in traditional CSV formats. 
The same flexbility applies to data export.
By providing standardization of record types, we can also perform broad, meta-searches across the library, for example for all videos containing participants within a particular age range.

\subsection{Annotations}

% rog this section needs work.

Establishing a timeline for each session allows temporally-positioned annotations to be attached to the session.
Currently we support two types of annotations: structured annotations based on records and added by the data contributors; and simple annotations contributed by the community.
Eventually we hope to extend both of these to support the full range of annotations performed during analysis, both directly on the web and imported from existing software.

First, records can be attached not only to entire sessions, but also to arbitrary temporal sub-sets of a session.
If any record relates to a part, scene, or section of a session---for example a participant being present for only part of a recording---it can be applied within the timeline of a session to only that appropriate part.
This allows an additional type of organizational structure researchers employ, wherein records define virtual sub-sessions that can be grouped and interacted with as units, without needing to cut up source videos.
This can also theoretically be used to represent annotations at a finer temporal scale, for example individual trials or events that are normally handled by off-line annotation software.
However, the appropriate interfaces and tools to input or import these annotations are still being developed.

Second, we support simple keyword-based tagging and comment-based discussions on any temporal segment of a session.
Such annotations can also be added through our API by automated video analysis according to features of interest identified in the data.
In addition to enriching datasets and enabling dialog between researchers, we expect these annotations to provide an index target for additional discovery capabilities for users looking for particular features within data \cite{Lanagan_Smeaton_2012}. Where research of community tagging provides evidence for the case that not many users actually do participate in tagging, we intend to leverage automated tagging through already existing metadata and analysis to seed a base of tags for reuse by the community \cite{Yang_Lu_Giles_2011, Farooq_etal_2007}.

\subsection{Technical infrastructure}

Databrary is a new, cohesive web application, built in Scala on the Play Framework\footnote{http://playframework.com/} to enable a responsive user interface, a complete API, and high-performance streaming.
Recordings and other files are placed in content-based filesystem storage, and all structured data are stored in a PostgreSQL database, leveraging its geometric indexing capabilities for temporal data.
All uploaded recordings are automatically transcoded to a standard format to enable cross-platform HTML5-based streaming and downloading for off-line access, currently H.264/AAC in an MPEG-4 container for video.
This transcoding utilizes the high performance computing cluster at the host campus of New York University, using ffmpeg's libav bindings for both this and direct access to video frames and clips.
The user interface is built primarily on the Angular web framework\footnote{http://angularjs.org/}, and all data access is performed through an open JSON API.
Although hosted data are protected, all source code is released under a GPLv3 license on github\footnote{http://github.com/databrary}.

\section{Repository}

The Databrary website accepted contributions beginning in early 2014 and opened for general use in October 2014.
It currently hosts 5,700 video files totaling 1,600 hours of recordings along with 2,200 additional files.
These files comprise 2,400 sessions and are covered by 1,300 records (including 1100 individual participants).
Data originates from 35 individual contributors across 25 different universities.

\subsection{Normalization}

Since we primarily deal with video and audio data, standardizing on an encoding format allows for convenient, consistent access without much loss of flexibility, while still allowing arbitrary resolutions, encoding qualities, and sample/frame-rates.
We also allow allow users, using certain portable formats (flat-files, documents, images), to include any data that cannot be encompassed by our standard formats.
In this way, we can incorporate data which may be relevant for re-use but not broadly adopted by the community, without making the system too rigid.

Here we have also introduced a number of metadata concepts where this tension becomes most apparent, namely the available record types, variables (and values within categorical data types), and tag keywords.
We have adopted an approach that encourages adherence to existing standardized options but  allows new types to be created as the data require it.
Initially we have achieved this through manual review and off-line curation of contributions, dynamically extending the system as new types of data were discovered.
For many previously collected datasets, this has proven the most effective method to learn from data and avoid burdening contributors.
However, since we expect the bulk of the data in the library to come from newly collected sources (due largely to privacy constraints on existing datasets), we have additionally taken a distinct, user-driven approach.

We found that while some researchers in our target domains have standardized data management practices, a large majority have no special data management or collection tools, often using a combination of hand-written paper, simple spreadsheets, existing video annotation tools, statistical analysis software, and available filesystem storage.
This suggested the opportunity to supply a cohesive data management platform where all collected raw data, including recordings, contextual information, and measurements, could be entered, stored, and exported for later off-line analysis.
Providing a targeted user interface to meet the existing and evolving needs of our contributed data allows us to control the means of data entry and thus achieve a greater amount of normalization in the data from the outset through liberal use of auto-completion, suggestions, and drop-down options.
With the considerable benefit of reducing storage requirements within researcher's labs, and often adding convenience over existing practices, this has proven an attractive prospect for researchers.

Achieving a sustainable balance will no doubt be challenging, but by using the above approaches and limiting our focus to particular research domains we have been able to achieve substantial standardization across the entire data library without overly burdening contributors.

\subsection{Discovery}

While increased normalization facilitates filtering and targeted searches for data, the general problems of browsing and discovery remain largely unsolved.
This problem is compounded for us because we expect researchers not only to reuse and reanalyze data within their own research area, but potentially to find new uses data originally collected for entirely different purposes.
Recordings with potential value in one research area may be primarily labeled with contextual information from a different research area.
Ultimately, making this determination requires direct, human observation of the recordings, and so a large part of solving the discovery problem will be simply to present researchers with short, representative samples from a variety of datasets.

Another important source for search targets would come from annotations supplied both by the original researchers during analysis as well as by the community.
In the former case, normalization remains a concern, as most annotation schemes researchers employ involve opaque, symbolic or numeric codes with little semantic content.
We are currently working to support import and export workflows for existing annotations tools that will be able to add meaningful definitions to these codes while researchers continue their current practices.
Beyond that, we also plan to build a new, more efficient and structured annotation web interface enabling more of the analysis workflow directly within our system.
In this way, Databrary will allow researchers to work with their recordings directly in raw form without having to transfer or transform them, while also capturing more meaningful and standardized contextual information from this process.

\section{Conclusions}

Starting from the perspective of a general-purpose repository, we have introduced a number of  restrictions to create one specialized for specific types of data.
By focusing on video and audio recordings we can normalize presentation and discovery of data across the site, and also introduce temporal structure to data and annotations driven by the nature of these recordings.
By limiting both file and metadata organization to a flat (but overlapping) set of sessions and records, we capture session-based research data within a cohesive and intuitive user interface.
Although we allow additional data files, we encourage users with other types of rich data, such as physiological recordings, to store and link to those resources externally.
Similarly, by targeting reuse of existing research data, rather than replication or meta-analysis, we limit the scope of accepted data to the original recordings and early analysis phases that generate the most value.

Having made significant progress building ingest and data entry services that have let us grow the library, we now must turn our efforts towards discovery and reuse, finding ways to better describe and elaborate on collected data.
To do this we look both to the original data contributors for ways that their existing analysis workflows can generate usable content descriptions, and to the community of researchers to label and comment on datasets as they explore.
Thus we seek to incorporate these existing annotation processes into Databrary, both to capture and define these annotations as generalizable search terms, and to make these workflows more convenient by obviating existing file transfer and transformation burdens.
Ultimately we hope to allow and capture instances of data reuse in this same way, so that content descriptions are aggregated from multiple annotation passes from different users and existing data can continuously increase in value for researchers.

There is also active discussion around the role of repositories and libraries in archiving and providing access to research data, where the roles and responsibilities between libraries and research departments is not entirely established \cite{Castelli_etal_2013, Nielson_Hjørland_2014, Macmillan_2014, Pinfield_etal_2014}.
Although resolving this complicated issue is not directly within our purview, we have laid out a functioning system for research data management that may bring data closer established library standards that achieve these goals.
We may be well positioned to provide interoperability with library-based metadata schemas (such as export of data packages cross-walked to Dublin Core or METS schemas) and reach OAI-PMH compliance, so as to automatically incorporate the data that researchers add to Databrary into federated library searches with other domain specific data repositories.
By providing a refined API and assigning Digital Object Identifiers (DOIs) to volumes, we can also allow libraries and other information systems to tap into Databrary datasets.

Our single-minded focus on making collected, identifiable recordings of human research available and reusable to a whole community of researchers gives us a novel perspective on building domain-specific research repositories.
The success we've had in engaging, involving, and extracting data from researchers is due largely to this focus, but it has come at the cost of limitations in scope and difficulties of interoperability and discovery that many more traditional repositories avoid.
By slowly growing our scope and relaxing some of these limitations, we hope to leverage more of the established knowledge of the digital library community in order to make this repository more globally applicable to the storage and accessibility of research data.

\section*{Acknowledgments}

This work was supported by the NSF (BCS-1238599) and the NICHD (U01-HD-076595-01).
The authors gratefully acknowledge the NYU Libraries for their valuable advice and consultation.

\bibliographystyle{abbrv}
\bibliography{references}

\end{document}
