% This is "sig-alternate.tex" V2.0 May 2012
% This file should be compiled with V2.5 of "sig-alternate.cls" May 2012
%
% This example file demonstrates the use of the 'sig-alternate.cls'
% V2.5 LaTeX2e document class file. It is for those submitting
% articles to ACM Conference Proceedings WHO DO NOT WISH TO
% STRICTLY ADHERE TO THE SIGS (PUBS-BOARD-ENDORSED) STYLE.
% The 'sig-alternate.cls' file will produce a similar-looking,
% albeit, 'tighter' paper resulting in, invariably, fewer pages.
%
% ----------------------------------------------------------------------------------------------------------------
% This .tex file (and associated .cls V2.5) produces:
%       1) The Permission Statement
%       2) The Conference (location) Info information
%       3) The Copyright Line with ACM data
%       4) NO page numbers
%
% as against the acm_proc_article-sp.cls file which
% DOES NOT produce 1) thru' 3) above.
%
% Using 'sig-alternate.cls' you have control, however, from within
% the source .tex file, over both the CopyrightYear
% (defaulted to 200X) and the ACM Copyright Data
% (defaulted to X-XXXXX-XX-X/XX/XX).
% e.g.
% \CopyrightYear{2007} will cause 2007 to appear in the copyright line.
% \crdata{0-12345-67-8/90/12} will cause 0-12345-67-8/90/12 to appear in the copyright line.
%
% ---------------------------------------------------------------------------------------------------------------
% This .tex source is an example which *does* use
% the .bib file (from which the .bbl file % is produced).
% REMEMBER HOWEVER: After having produced the .bbl file,
% and prior to final submission, you *NEED* to 'insert'
% your .bbl file into your source .tex file so as to provide
% ONE 'self-contained' source file.
%
% ================= IF YOU HAVE QUESTIONS =======================
% Questions regarding the SIGS styles, SIGS policies and
% procedures, Conferences etc. should be sent to
% Adrienne Griscti (griscti@acm.org)
%
% Technical questions _only_ to
% Gerald Murray (murray@hq.acm.org)
% ===============================================================
%
% For tracking purposes - this is V2.0 - May 2012

\documentclass{sig-alternate}

\begin{document}
%
% --- Author Metadata here ---
\conferenceinfo{JCDL}{'15 Knoxville, Tennesse USA}
%\CopyrightYear{2007} % Allows default copyright year (20XX) to be over-ridden - IF NEED BE.
%\crdata{0-12345-67-8/90/01}  % Allows default copyright data (0-89791-88-6/97/05) to be over-ridden - IF NEED BE.
% --- End of Author Metadata ---

\title{Databrary: A digital data library for sharing research video}

\numberofauthors{6}

\author{
\alignauthor
Rick O. Gilmore, Ph.D.\\
	\affaddress{The Pennsylvania State University}\\
	\affaddress{Department of Psychology}\\
	\affadddress{University Park, PA 16802}\\
	\email{rogilmore@psu.edu}

\alignauthor
Lisa A. Steiger\\
	\affaddress{Databrary Project}\\
	\affaddress{New York University}
	\affaddress{196 Mercer Street, Room 807}\\
	\affaddress{New York, NY 10012}\\
	\email{lisa@databrary.org}

\alignauthor
Dylan A. Simon\\
	\affaddress{Databrary Project}\\
	\affaddress{New York University}
	\affaddress{196 Mercer Street, Room 807}\\
	\affaddress{New York, NY 10012}\\
	\email{dylan@databrary.org}

\and

\alignauthor
Drew Gordon\\
	\affaddress{Databrary Project}\\
	\affaddress{New York University}
	\affaddress{196 Mercer Street, Room 807}\\
	\affaddress{New York, NY 10012}\\
	\email{drew@databrary.org}

\alignauthor
David S. Millman\\
	\affaddress{New York University}
	\affaddress{New York, NY 10012}\\
	\email{dsm@nyu.edu}

\alignauthor
Karen E. Adolph, Ph.D.\\
	\affaddress{New York University}
	\affaddress{Department of Psychology}
	\affaddress{New York, NY 10012}\\
	\email{karen.adolph@nyu.edu}

}

\maketitle

\begin{abstract}

Video captures the complexity, richness, and diversity of behavior
unlike any other tool. As a result, large numbers of researchers who
study human or animal behavior employ video. Video documents itself to a
large degree, and therefore has significant potential for data re-use.
Still, researchers rarely share video. Video often contains information
about personal identities, so considerations about research ethics pose
challenges to sharing. The relatively large size of video files and
diversity of formats pose technical challenges. In this paper, we will
describe how the web-based Databrary data library has overcome the most
significant barriers to sharing video within the developmental and
learning sciences community, including solutions to maintaining
participant privacy, data tagging, and data management.
\end{abstract}

**Categories and Subject Descriptors**

\category{H.2.8}{Database Management}{Database Applications}[image
databases, scientific databases]

\terms{Management, Documentation, Design, Security, Legal Aspects}

\keywords{Data sharing, open science, video, psychology, developmental science}

\section{Introduction}

Permission to make digital or hard copies of all or part of this work
for personal or classroom use is granted without fee provided that
copies are not made or distributed for profit or commercial advantage
and that copies bear this notice and the full citation on the first
page. To copy otherwise, or republish, to post on servers or to
redistribute to lists, requires prior specific permission and/or a fee.

Film and video recording have been a mainstay of research in the
developmental sciences for nearly 100 years [CITE]. A typical lab
acquires, on average, 12 hours of video per week depicting behavior in
natural or laboratory environments [CITE]. These recordings constitute a
primary source of raw data. Prior to descriptive or statistical
analyses, researchers score or “code” recordings, mostly manually, using
commercial software or makeshift tools. However, even the most detailed
and laborious video coding does not exhaust the possibilities for
analyses by other researchers with different questions. Thus, sharing
video promises a wide range use cases, perhaps more than sharing
flat-file data or static images. Despite the wealth of untapped
information available in research videos and the labor expended in
collecting and coding them, developmental researchers rarely share data.
In the prevailing culture, investigators conduct research privately
within their own laboratories. Data are destroyed or molder away on a
shelf after the research is published. The Databrary project aims to
change the status quo, making sharing standard practice.

Databrary enables researchers to share video and associated flat-file
data for research and educational purposes. For example, a researcher
interested in language development might reuse a set of video recordings
created originally to study the development of walking. Researchers can
conduct integrative analyses by combining data across disparate studies,
increase sample sizes, verify the feasibility of proposed projects,
compare methods, learn about procedures, and independently verify
results. Similarly, investigators can reuse shared videos for
educational purposes such as to find illustrative examples for teaching,
use exemplars in presentations, show students how to conduct procedures,
or provide supplemental materials for reviewers.

Data sharing and open science practices are emerging or accepted norms
in the biological, physical and earth sciences with demonstrated
benefits for scientific transparency and accelerated discovery [1-4].
Open research practices are gaining traction in the behavioral and
social sciences [5-10]. However, open data sharing among developmental
researchers is still the exception [11] not common practice. In seeking
to expand sharing of video-based data, Databrary faces challenges
concerning the interplay among institutional research policies, ethical
issues related to obtaining participants’ consent for sharing, and
technical authorization systems. This paper describes the current state
of the Databrary project including the project organization, how content
is acquired, stored, preserved, and accessed, policy considerations, and
how the library has affected and hopes to affect the behavioral science
research community.

\section{Project Organization}

Databrary is a joint project of New York University (NYU) and The
Pennsylvania State University (Penn State). The Databrary project began
with a workshop on open video data sharing funded by the National
Science Foundation (NSF) under Grant No. BCS-1139702 at which
developmental scientists, computer scientists, library scientists, and
federal agency program officers discussed the promises and challenges of
sharing video data. NSF and the National Institute of Child Health and
Human Development (NICHD) have provided funding under grant No.
BCS-1238599 and Cooperative Agreement U01-HD-076595, respectively. The
interdisciplinary team bring expertise in developmental and neural
science, information technology, library science, software development,
data curation, and community engagement. In addition, a board of expert
advisors composed of developmental scientists, library scientists, and
leaders of data repositories in the behavioral and social sciences
provide guidance.

The project is organized around five core activities.

\subsection{Curation}

To seed the Databrary, staff identified and selected video datasets for
manual curation. The datasets were selected based on availability,
presumed interest to developmental science community, suitability for
the organization and file structure of the repository, diversity of
study type, and on the availability of appropriate sharing permissions
(see 2.2 below). Manual curation of archival datasets continues under
the direction of a staff member with expertise in library and
information sciences.

Nevertheless, we have found manual data curation time consuming and
difficult, especially for researchers. The more difficult and
time-consuming data sharing becomes, the less likely it will be that
researchers participate despite best intentions. So, we have created
user-friendly spreadsheet-like interfaces that enable researchers to
self-curate datasets as they are collected. At the end of a data
collection session, researchers may upload videos and other data files
collected along with metadata about participant demographic
characteristics or experimental conditions. Databrary transcodes videos
to a standard format in the background, and the system alerts the user
when the video can be previewed. Researchers may also upload study-level
materials -- code books, flat data files used for statistical analyses,
images of experimental equipment, links to related external information
(e.g., GitHub repositories), etc.

This scheme, which we call "upload-as-you-go", makes the data curation
required for sharing part of the normal scientific workflow, not a
burdensome after thought. No data are shared until the researcher grants
access, but unlike the *ad hoc* organization in a typical investigator's
laboratory, on Databrary, the data are organized in a standardized way
that could be readily and easily shared with others when the
investigator is ready to do so.

Sharing video requires special considerations because the data cannot be
de-identified without reducing its value. No previously established
research data repository has specialized in video data sharing.
Therefore, Databrary had to develop a new policy and permission
framework for sharing identifiable and sensitive data while
simultaneously preserving the rights of research participants.

The framework has two components, both of which build on
well-established principles. One extends the idea of informed consent to
participate in research by requiring that participants give explicit
permission for their identifiable or sensitive data to be shared with
other researchers. Databrary makes available on its website template
language researchers may use for seeking permission to share, along with
video demonstrations and additional guidance. So, researchers who wish
to share video must either use the template or demonstrate that
participants have granted an equivalent level of permission. The second
component involves limiting access to individuals who have prior formal
authorization. Researchers must either secure authorization to access
Databrary from their institution, typically a university where they
conduct independent research, or must have another authorized researcher
grant them access. Institution-level authorization requires a formal
agreement with Databrary. Fundamentally, the Databrary framework ensures
that participants control whether other researchers can view their video
(or other identifiable data), and only researchers who pledge to uphold
ethical principles, including maintaining participant confidentiality,
may access Databrary's sensitive materials.

Databrary developed these policies and agreements in close consultation
with Institutional Review Boards (IRBs), sponsored research privacy
administrators, and legal staff at NYU and Penn State, with other
experts in research data ethics, and with Federal officials. See
http://databrary.org/access/guide.html for more details.

The centerpiece of the Databrary project is the repository itself. Our
team of programmers has developed digital library software, including
asset management, workflow, access control, discovery services, and a
user interface. This work is closely coordinated with the central NYU IT
infrastructure providers for services such as high-capacity storage,
high-security storage, and high-performance computing (for video
processing). The long-term goal is to build the cyber-infrastructure for
seamless search, streaming, uploading, and downloading and to provide an
open-source framework for other entities to build on the Databrary
software or extend it to other content domains.

\subsection{Annotation and Metadata}

Video becomes most useful for conducting research when it is transformed
into quantitative and qualitative data in a flat-file format. Thus, the
Databrary team is enhancing a general-purpose video-coding software
application specialized for exploring, annotating, tagging, scoring, and
visualizing the video files in preparation for qualitative and
quantitative analyses. Toward that end, we have released Datavyu
(datayvu.org), a free, open source, multi-platform desktop application.
Datavyu is a fork of the OpenSHAPA tool used by some developmental
researchers. The long-term goal is to build a web-based version of
Datavyu that will provide both data analysis and data management
services required for preparing and organizing video material prior to
its storage in Databrary.

\subsection{Community Practices}

The current zeitgeist in the developmental science community is one of
privacy and data hoarding. A primary activity of the Databrary project
is to introduce the practice of open data sharing to the developmental
science community. We are doing so through announcements to
developmental research societies, colloquia, and conference
presentations. The team participates in workshops to discuss sharing
issues and hosts events to teach potential users about sharing and
analysis tools. We post links on websites frequented by developmental
researchers and participate in user forums. In addition, we have asked
leaders in developmental science to set an example for other researchers
by publicly committing to open video data sharing. More than a hundred
have responded to the call
(http://databrary.org/community/contributors.html). Our long-term goal
is a self-sustaining community of researchers committed to open
science.

\section{Building the Library}

Existing data management practices pose challenges for video sharing.
Despite similar research methods, study designs vary widely. No two
developmental science labs manage data in the same way. Some studies are
longitudinal, where researchers observe the same participants at
multiple sessions. Some are cross-sectional, where researchers observe
each participant at only one session. The timing of observations may be
determined by participants’ age (e.g., 4-month-olds, 8-month-olds,
12-month-olds) or abilities (e.g., preverbal, one-word utterances,
sentences), experimentally determined variables (e.g., pre- and
post-intervention), or other factors. Some labs organize longitudinal
data by grouping files first by participant and then by session date.
Other labs organize longitudinal data by session and then by
participant. Still others organize longitudinal data based on task (book
reading, block building, free play). Some researchers institute a
central data management system to be followed by all the lab members,
providing easier access and greater transparency for the entire lab but
not necessarily providing the structure for similar benefits to
researchers unfamiliar with the lab’s practices. Other researchers allow
their students to keep separate records, making it difficult to share
data even within the lab. Some researchers keep videos, metadata, and
analyses together, and some do not. Idiosyncratic terms, record-keeping,
and data management systems are the norm. Databrary must enable
researchers to discover and understand each other’s materials,
regardless of the original investigator’s data management system.

Databrary takes several approaches to address this challenge: A flexible
data model, informal metadata contributions by users via tagging and
commenting, mediated curation, and incentives to deposit materials more
consistently. Figure 1 illustrates the principal data model. It
accommodates different hierarchies and organizations, such as those
described in the examples above.

The full expressive flexibility of the data model is not evident to
Databrary users. Instead, users see different aspects of the data model
at different times, depending on the context. Researchers organize their
materials by acquisition date and time into structures called sessions.
A session corresponds to a unique recording episode and contains one or
more recordings or related flat files (assets). Researchers can group
sessions in different ways, using whatever groupings are appropriate for
the particular dataset. These groupings may also identify particular
time segments of a recording to distinguish tasks, events, or
participants that comprise the session. In the data model, the aggregate
of these flexible groupings is called a volume. In practice, researchers
combine sessions or segments of sessions within and across datasets to
form the raw material that are subsequently described in published
articles and presentations. Thus, Databrary contributors can combine
sessions or segments within and across datasets with ancillary
materials, such as coding manuals, Datavyu spreadsheets, statistical
analyses, questionnaires, IRB documents, computer code, sample displays,
and links to published journal articles in an aggregate structure
investigators call a study. Like datasets, studies are represented as
volumes in the data model.

The discovery and browsing interfaces offer content items at the volume
(study or dataset) level to users searching the library. The
contributor-assigned groupings (records), which may include relevant
metadata about participants (measures, such as participant demographic
information, domain-specific survey information, location data,
condition variables, task descriptions, or other properties), allow a
second level of filtering and organization within and between relevant
identified studies or datasets. This enables users to quickly identify
data that may be relevant to their own interests or research.

Authorized Databrary Investigators can also add keyword tags at the
study/dataset, session, or segment level, and can endorse or deprecate
keywords that have been suggested by other authorized researchers. These
abilities are critical because different researchers may have very
different reasons for citing a study or using data. Future
implementations may extend the ability of authorized researchers to
annotate studies with other kinds of metadata.

The Databrary team recognizes that other data repositories enforce
strict metadata ontologies and that doing so may have benefits in
sub-domains of research when there is community consensus [12]. We will
support standard data coding ontologies among researchers, but only
enforce standardization for a small set of standard tags such as study
date, participant birth date, and sex. We will also encourage
contributors to report race, ethnicity, primary language, language of
dataset, and location of session. The Databrary team chose not to
require strict metadata ontologies for several reasons. We hope to
reduce the pre-deposit curational demands on contributors, encourage the
repurposing of shared data, and foster the rapid adoption of data
sharing. Developing and achieving agreement on metadata ontologies can
take significant amounts of time. Video data are so rich and complex
that in many domains, researchers have not settled on standard
definitions for particular behaviors and may have little current need
for standard tasks, procedures, or terminology. However, we will also
encourage users to re-use tags and terminology by suggesting common or
similar terms, without confining users to these suggestions. User
communities within Databrary may eventually converge on common
conceptual and metadata ontologies based on the most common (and
commonly endorsed) keyword tags, but standardized ontologies are not
necessary for browsing and searching in most of the use cases we
envision.

Our experiences curating and ingesting archival datasets have
highlighted the considerable value of contributors entering raw video
data into Databrary as soon as recordings are acquired. Immediate
uploading reduces the workload on investigators, minimizes the risk of
data loss and corruption, and accelerates the speed with which materials
become openly available. To encourage immediate uploading, Databrary
provides a complete set of controls so that researchers can restrict
access to their data to only their own labs or to other users of their
choosing. Datasets can be shared at a later point when data collection
and ancillary materials are complete, whenever the contributor is
comfortable sharing, or when journals or funders require it. Databrary
has published a Data Sharing Manifesto [13] that explains to researchers
the Databrary philosophy. Standards about when data should be shared are
evolving. Our philosophy is consistent with concepts and practices in
other domains where data sharing is the norm (e.g., www.iedata.org).
Planned enhancements to the Datavyu tool will allow contributors to
organize video files and metadata as part of the analysis process. This
will facilitate the contribution of packaged datasets to Databrary.

As part of the curation process, Databrary stores at least two versions
of each item of Databrary video content: a copy for access, and the
received original file if it was digital, or a 10-bit YUV digital
preservation copy if the original version was not digital. Currently,
the access version format is H.264 (HiP) with AAC audio in an MPEG-4
container, although we expect the appropriate video formats to change
over time, as has been the case with many digital video formats in
recent years.

For preservation, the original file (if digital) or the preservation
copy will be stored in a long-term preservation repository managed
jointly by the NYU Libraries and the central Information Technology
Services unit. This repository ensures that each content item has a METS
[14] structural metadata file that associates the digital asset with its
metadata. It stores files in two mirrored and geographically distributed
locations, and a third copy on offsite tape; it performs regular fixity
checks; and it provides a format migration capacity, in the event that a
stored format becomes at risk of obsolescence.

\section{Policies Framework}

Sharing video recordings poses a unique challenge to existing data
sharing policies because videos contain personally identifying
information—specifically participants’ faces and voices and often the
insides of their homes and classrooms. Sharing personally identifiable
information puts research participants and others depicted in recordings
at increased risk for loss of privacy. At the same time, blurring or
altering original recordings to hide identities undermines or eliminates
their value to other researchers. Often, participants’ faces and voices
produce the behaviors of interest. So, Databrary has elected to maximize
the potential for data re-use by keeping recordings in their original
unaltered form. Instead of removing participants’ identities, Databrary
restricts access to identifiable or sensitive data to authorized
researchers. Further, Databrary provides access only when the people
depicted have given permission for their information to be shared with
other researchers.

\subsection{Restricting Access}

Databrary provides access to shared data only to authorized researchers
who have agreed to uphold common practices concerning the responsible
and ethical use of identifiable and sensitive data. To become an
authorized investigator, applicants must register on the site and
electronically sign the Authorized Investigator Agreement, which must
also be co-signed by the applicant’s institution. Full privileges will
be granted only to researchers with principal investigator (PI) status
at their institutions. Other researchers may be granted privileges if
they are affiliated with a PI who agrees to sponsor and supervise their
application. Initially there will be a manual process to identify the
institutional representative—typically the authorizing official of the
university—who can co-sign the Investigator Agreement. However, as the
user groups at each university expand, Databrary may implement
administrative accounts at each institution. This will enable the
authorizing official to independently manage the authorizations of
individual researchers at her institution.

\subsection{Seeking permission to share identifiable data}

Data from a particular session may be stored in Databrary for the
contributing researcher’s use whether the records are shared with other
scientists or not. When a researcher chooses to share, Databrary makes
records openly available to the community of authorized researchers only
if the people depicted in the recordings have given permission to
release the data for sharing. Thus, Databrary requires that people
depicted in recordings grant permission before their information can be
shared. Databrary’s policies extend currently accepted principles of
informed consent to the situation where participants are granted
authority to consent to (or refuse) the release of their identifiable
data.

We developed these ideas in close collaboration with the NYU and PSU IRB
staff. To formalize the process of acquiring permission, we developed a
Participant Release Form Template, based on photo or video release
language many researchers use currently. The template release form has
standard language that Databrary recommends investigators should use
with study participants. This language makes it easy for participants to
understand what is involved in sharing their video data, with whom it
will be shared, and the potential risks associated with releasing their
video and other identifiable data to other researchers. Use of the
template also allows for the standardization of language associated with
the release of identifiable or sensitive participant data.

\subsection{Technical assistance with IRBs}

Some IRBs may deem an investigator’s existing, approved video or photo
release form equivalent to the Databrary release. This enables a
researcher to share with Databrary recordings they have already
collected. However, most researchers will need to modify their research
protocols, by adding the Databrary sharing permission procedures, prior
to collecting new shareable video data. Databrary staff are available to
advise potential data contributors about how to amend existing research
protocols so that the information acquired is Databrary-compliant.
Protocol amendments involve seeking approval for use of the Databrary
template release form and modifying the time period over which collected
data will be made available. Specifically, researchers must remove any
clauses in research consent documents that require data destruction
after some fixed period of time since Databrary intends to store shared
data indefinitely.

\section{Impact}

The Databrary project has already begun to have an impact. As of
mid-January 2015, more than 57 scientists from 35 institutions in North
America, the UK, and Europe have received authorization for full access
to and data sharing with Databrary. More than 50 additional researchers
are in the process of securing institutional approval. In consultation
with Databrary staff, many of these researchers have secured or are in
the process of securing permission from their research ethics boards to
share archival or new data. Thus, the Databrary project has laid the
groundwork for change in scientific culture around data sharing within
the community of developmental science researchers who are most familiar
with it. In a similar vein, Databrary is attempting to lead by example
where open science practices are concerned. Databrary is an open source
project. The entire code base is available on Github
(github.com/databrary) as are all policy documents. The project team
consults regularly with other leaders in the open science, data sharing,
and data repository communities, and we share best practices among them.

Databrary has also begun to have an impact in the policy arena. The
notion that identifiable research data may be shared – under the right
circumstances – is not new to Databrary [15,16]. But, Databrary has
created a set of policies and template documents that will help IRBs to
come to see that seeking permission to share data is merely an extension
of the principle of informed consent. Furthermore, Databrary’s
Authorized Investigator Agreement combines provisions for data
contribution with those guiding data use, part of an overall effort to
reduce barriers to sharing and re-use. Our colleagues in the data
sharing community tell us that this combination represents an innovation
in itself.

The Databrary team has come to understand that laboratory data
management practices pose significant challenges to widespread data
sharing. Simply put, many researchers deploy workflows that would
require significant modification in order for video files and associated
metadata to be readily shared. In response, Databrary has expanded its
data curation expertise and capacity. However, hand-curating significant
volumes of research data in this way is not sustainable in the long
term. Building tools that can enable self-curation will be essential to
the future success and sustainability of Databrary. As a result, the
team has begun to compile best practices for data management and
curation that will be folded into future data coding (Datavyu) and data
library (Databrary) features. We note that the process of understanding
the diversity of data management techniques in developmental science
poses unique opportunities for librarians and curators to learn how
researchers organize and manage their data for daily use. It also
affords researchers opportunities to learn from their library science
colleagues about the value of adopting best practices in data
organization and management.

\section{Conclusions}

The Databrary project aims to increase scientific transparency and
accelerate discovery in developmental science by creating a
user-friendly and powerful infrastructure for researchers to share video
and related data. Clearly, sharing video data poses technical and policy
challenges, but it presents significant opportunities for accelerating
discovery if these challenges can be met successfully. The Databrary
project has already made significant strides in identifying and
overcoming many of the obstacles, and the tools and infrastructure that
we develop promise to enhance data sharing and management practices in
the entire behavioral science community.

\section{Acknowledgments}

This work was supported by the National Science Foundation (BCS-1238599)
and the National Institute of Child Health and Human Development
(U01-HD-076595-01). The authors gratefully acknowledge the NYU
Libraries, Human Connectome Project, the Personal Genome Project, the
Inter-university Consortium for Political and Social Research, the
Center for Open Science, Dataverse, Data Dryad, and staff from the
National Database for Autism Research for their valuable advice and
consultation.

REFERENCES
==========

Curry A (2011) Rescue of Old Data Offers Lesson for Particle Physicists.
Science 331:694-695. doi:10.1126/science.331.6018.694

Integrated Earth Data Applications. (2014). http://www.iedadata.org.
Accessed 10 February 2014

Kaye J, Heeney C, Hawkins N, de Vries J, Boddington P (2009) Data
sharing in genomics [mdash] re-shaping scientific practice. Nat Rev
Genet 10 (5):331-335. doi:10.1038/nrg2573

Overpeck JT, Meehl GA, Bony S, Easterling DR (2011) Climate Data
Challenges in the 21st Century. Science 331 (6018):700-702.
doi:10.1126/science.1197869

Lunshof J, Church G, Prainsack B (2014) Raw Personal Data: Providing
Access. Science 343 (6169):373-374. doi:10.1126/science.1249382

Adolph K, Gilmore RO, Freeman C, Sanderson P, Millman D (2012) Toward
Open Behavioral Science. Psychological Inquiry: An International Journal
for the Advancement of Psychological Theory 23 (3):244-247.
doi:10.1080/1047840X.2012.705133

King G (2011) Ensuring the Data-Rich Future of the Social Sciences.
Science 331 (6018):719-721. doi:10.1126/science.1197872.

Association for Psychological Science (2012) Special Section on
Replicability in Psychological Science: A Crisis of Confidence.
Perspectives on Psychological Science 7 (6)

Nosek BA, Bar-Anan Y (2012) Scientific Utopia: I. Opening Scientific
Communication. Psychological Inquiry 23 (3):217-243.
doi:10.1080/1047840X.2012.692215

Nosek BA, Spies JR, Motyl M (2012) Scientific Utopia: II. Restructuring
Incentives and Practices to Promote Truth Over Publishability.
Perspectives on Psychological Science 7 (6):615-631.
doi:10.1177/1745691612459058

MacWhinney B (2001) From CHILDES to TalkBank. In: M. Almgren AB, M.
Ezeizaberrena, I. Idiazabal & B. MacWhinney (ed) Research on Child
Language Acquisition. Cascadilla, Massachusetts, pp 17-34

Cognitive Atlas. (2014). http://www.cognitiveatlas.org. Accessed 10
February 2014

Databrary (2013) Databrary Data Sharing Manifesto.
http://databrary.org/user-guide/policies/data-sharing-manifesto.html.
Accessed 10 February 2014

METS (2014) XSLT-based Metadata Encoding and Transcription Standards
Viewers. http://dlib.nyu.edu/metstools/metsviewer/. Accessed 10 February
2014

HCP (2014) Human Connectome Project.
http://www.humanconnectomeproject.org. Accessed 10 February 2014

PGP (2014) Personal Genome Project. http://www.personalgenomes.org.
Accessed 10 February 2014

\end{document}
