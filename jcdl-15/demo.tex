\documentclass{sig-alternate}

\begin{document}
\conferenceinfo{JCDL}{'15 Knoxville, Tennessee USA}

\title{Databrary: Enabling sharing and reuse of research video}

\numberofauthors{4}

\author{
\alignauthor
Dylan A. Simon\\
	\affaddr{Databrary Project}\\
	\affaddr{New York University}\\
	\affaddr{New York, NY, USA}\\
	\email{dylan@databrary.org}
\alignauthor
Andrew Gordon\\
	\affaddr{Databrary Project}\\
	\affaddr{New York University}\\
	\affaddr{New York, NY, USA}\\
	\email{drew@databrary.org}
\alignauthor
Lisa Steiger\\
	\affaddr{Databrary Project}\\
	\affaddr{New York University}\\
	\affaddr{New York, NY, USA}\\
	\email{lisa@databrary.org}
\and
Rick O. Gilmore\\
	\affaddr{Department of Psychology}\\
	\affaddr{Penn State}\\
	\affaddr{University Park, PA, USA}\\
	\email{rogilmore@psu.edu}
}

\maketitle

\begin{abstract}
Video and audio recordings serve as a primary data source in many fields, especially in the social and behavioral sciences.
Recordings present unique opportunities for reuse and reanalysis for novel scientific purposes, but also present challenges around respecting the privacy of individuals depicted.
Databrary is a web-based service for sharing and reusing the video data created by researchers in the developmental and learning sciences.
By investigating how researchers organize, analyze, and mine their own recordings, we have implemented a system that empowers researchers to capture, store, and share recordings in a standardized way.
This demo will provide a tour through the Databrary service, highlighting how it promotes storage, sharing, and management for research data, controls user access to restricted human subject data, and facilitates browsing and discoverability of datasets.
\end{abstract}

\category{H.2.8}{Database Management}{Database Applications}[scientific databases, image databases]
\category{H.3.5}{Information Storage and Retrieval}{Online Information Systems}[data sharing, web-based services]

\terms{Design, Standardization}

\keywords{Data sharing, open science, video, psychology, developmental science}

\break
\section{Introduction}

Video and audio recordings serve as a primary data source for scientific research in psychology, linguistics, education, anthropology, and many other disciplines.
They are, in contrast to most timeseries data, human-consumable and largely self-documenting, and are often analyzed without the use of sophisticated tools as human observers can directly glean much of the data's richness.
This allows recordings collected in one context to be used by others for different purposes with minimal explanation.

While video data have many benefits, they also contain highly identifiable aspects of depicted human participants: recordings contain participants' faces and voices, and their names are often spoken aloud.
Because of this, sharing video data presents challenges related to the privacy of human subjects.
In contrast to other forms of data that are non-identifiable or can be easily de-identified, video data cannot be easily de-identified without significantly diminishing its value and richness. 
Therefore, any solution for video sharing must provide a policy and access framework to ensure appropriate protection of research participants.

Despite their inherent reusability, researchers rarely share recordings with others, due largely to entrenched practices and perceived privacy requirements.
This lack of sharing prevents extracting the full value from recordings and impedes scientific discovery.
To address this problem, the Databrary project\footnote{https://databrary.org/} aims to build tools and policies that facilitate repurposing and reuse of recordings without placing undue burden on data contributors.
We have created a web-based data library to store and organize recordings and to capture contextual information necessary for reuse in standardized ways.

Our approach is to focus on video and audio recordings in a particular research domain, and build appropriate structured and unstructured resources around this core.
We designed the system by investigating how researchers studying human development and education organize their own data, deriving from those observations a unifying set of principles for organizing contextual metadata.
Thus, it facilitates the capture, standardization, discovery, and understanding of data to enable reuse of recordings at an unprecedented scale.

\section{Repository Details}

Databrary began accepting contributions in early 2014 and opened for general use in October 2014.
As of April 2015, it hosts 7,300 video files totaling 2,400 hours of recordings along with 2,600 additional files.
These files make up 3,100 study sessions and are covered by 2,000 metadata records (including 1,800 individual participants).
Data originate from 35 individual contributors across 25 different universities.

Databrary is an open-source web application\footnote{http://github.com/databrary/databrary}, featuring a responsive user interface, a JSON API, and high-performance video streaming.
Databrary stores at least two versions of each item of Databrary video content: a copy for access, and the received original file.
The access copy is generated by automatically transcoding the uploaded file to a standard format to enable cross-platform HTML5-based streaming and downloading for off-line access, currently H.264/AAC in an MPEG-4 container for video.

\section{Data Management}

Focusing on a particular research domain allows us to develop tools informed by how the intended community of users currently works, while accommodating the specific privacy concerns for data that often has a high disclosure risk.
We have created a set of data management features that empower researchers to actively manage their own projects---to upload data with accompanying metadata---as each study unfolds. 
These features include a spreadsheet interface for entering, editing, and viewing session-level metadata: participant demographics, conditions of study, tasks in the experiment, permission levels, study groups and conditions, etc. 
Most researchers use desktop spreadsheets for precisely this purpose in their own labs, making the interface and functionality intuitive to users. 
We have also implemented a timeline for uploading, viewing, and tagging video assets related to sessions.
The timeline view is designed to look and operate like video editing software commonly used in many research labs.
It allows users to upload video files, position them to reflect the temporal order of each component of data collection, and annotate video sections with user-generated tags. 

Providing a targeted user interface to meet the existing and evolving needs of our user base allows us to control and standardize the means of data entry and thus achieve a greater amount of normalization in the data from the outset, through liberal use of visualization, auto-completion, and suggestions.
In turn, this adds convenience and functionality for researchers over existing practices and alleviates storage burdens.

In addition, because video data involving human subjects is identifiable, we have built the access checks necessary to honor the participant consents under which the data were collected. 
User registration for gaining access to Databrary involves an authorization process by which site users are verified to be faculty, students, or researchers affiliated with an educational institution.
Users are required to review and sign an access agreement stating that they will treat the data they are able to access through Databrary in a manner that complies with the ethical handling of research data involving human subjects\footnote{http://databrary.org/access/policies/agreement.html}.
Finally, contributors are offered permission levels for files and groups of files that allow them to fine-tune who has access to their data. 

\section{Discovery and Reuse}

Databrary's mission is not merely the storage and management of research data, but also to make these data discoverable by other researchers and to facilitate their reuse.
Currently we support text searching over the entire repository, and we plan to continue to implement more refined searches that allows faceting of search results based on session metadata such as participant details, tasks involved in the experiment, study conditions, and context.
Additionally, Databrary allows for the annotation of videos and video segments with tags, which can serve as an index point for more refined search functionality.  

We aim to allow data to be reused in future studies, cited in scholarly communication, as well as used to aid in scientific and educational presentations.
To promote using datasets as a bibliographic source, each dataset comes with a formatted citation, a persistent URI, and soon will also be assigned a Digital Object Identifier (DOI) for referencing in scholarly publications and presentations.
Based on input from site users we have also built in the ability to excerpt or highlight clips from videos in such a way that they might serve as an exemplar clip for the dataset or be used in the classroom as an education tool or at professional conferences.

As we continue to gather feedback from our intended user base, we not only see Databrary as a tool that facilitates the standardized management of research data, but also as an opportunity to enhance the ability for academics to communicate about and augment their research.
We have already addressed the basic upload, data management, and storage needs of our target community, and have sufficiently answered the question of how to protect sensitive video data while providing wider opportunities to share those data.
Going forward, we hope to focus more on developing the tools for searching, discovering, and reusing those data, allowing researchers to work with their data in ways they had not previously imagined.

\section*{Acknowledgments}

This work was supported by the NSF (BCS-1238599) and the NICHD (U01-HD-076595-01).
The authors gratefully acknowledge the NYU Libraries for their valuable advice and consultation.

\end{document}
