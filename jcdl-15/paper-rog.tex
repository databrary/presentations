\documentclass{sig-alternate}

\begin{document}
\conferenceinfo{JCDL}{'15 Knoxville, Tennessee USA}

\title{Databrary: A digital data library for sharing research video}

\numberofauthors{4}

\author{
\alignauthor
Dylan A. Simon\\
	\affaddr{Databrary Project}\\
	\affaddr{New York University}\\
	\affaddr{New York, NY, USA}\\
	\email{dylan@databrary.org}
\alignauthor
Andrew Gordon\\
	\affaddr{Databrary Project}\\
	\affaddr{New York University}\\
	\affaddr{New York, NY, USA}\\
	\email{drew@databrary.org}
\alignauthor
Lisa Steiger\\
	\affaddr{Databrary Project}\\
	\affaddr{New York University}\\
	\affaddr{New York, NY, USA}\\
	\email{lisa@databrary.org}
\and
Rick O. Gilmore\\
	\affaddr{Penn State}\\
	\affaddr{Department of Psychology}\\
	\affaddr{University Park, PA, USA}\\
	\email{rogilmore@psu.edu}
}

\maketitle

\begin{abstract}
Video and audio recordings serve as a primary data source in many fields, especially in the social and behavioral sciences.
Recordings present unique opportunities for reuse and reanalysis for novel scientific purposes.
However, technical and policy challenges prevent existing recordings from being accessed and reused by other researchers.
By investigating how researchers organize, analyze, and mine their own recordings, we have implemented a data library that empowers researchers to capture, store, and share recordings in a standardized way.
Solutions developed to solve problems in Databrary's target scholarly domain have relevance to the digital library community.
\end{abstract}

\category{H.2.8}{Database Management}{Database Applications}[scientific databases, image databases]
\category{H.3.5}{Information Storage and Retrieval}{Online Information Systems}[data sharing, web-based services]

\terms{Design, Standardization}

\keywords{Data sharing, open science, video, psychology, developmental science}

\section{Introduction}

Video and audio recordings serve as a primary data source for scientific research in psychology, linguistics, education, anthropology, and many other disciplines.
Recordings are human-consumable and largely self-documenting. 
Most can be analyzed without the use of sophisticated tools. 
This contrasts with most forms of timeseries data.
% access in the sense of "what format is it in":
Further, recordings collected in one context may used for different purposes with minimal explanation.
This reduces the problems of documentation, provenance, and access that plague other data types and the repositories that store them. 
On the other hand, automated analysis of videos is an active but still immature research area, so the problems of search, discoverability, and meta-analysis over these media remain \cite{Albertson_2013, Lanagan_Smeaton_2012}.

Unfortunately, researchers rarely share recordings with others, due largely to entrenched practices and privacy requirements.
%something here or somewhere else about aversion (fear) to changing practices...
To address these problems, the Databrary project has developed policies and built tools that overcome barriers specific to the repurposing and reuse of recordings.
We designed the system by investigating how researchers organize their own data, deriving from those observations a unifying set of principles.
Our web-based data library empowers researchers to store and organize recordings in standardized ways without placing undue burdens on data contributors. 
We aim to strike a balance between general purpose and domain specific repositories and ultimately to enhance understanding and accelerate discovery.
In the sections that follow, we describe the system's architecture and infrastructure and briefly explain the rationale behind our design.
% previous para contains some of the ideas below...cut for space

% Research data repositories require 
% General purpose repositories permit unstructured storage of arbitrary files \cite{dataverse, dryad, purr}, while those with rigorously structured schema require contributions to be to carefully formatted \cite{HCP, PGP, talkbank}.
% The flexibility of general-purpose repositories make them especially well-suited to capture and archive entire datasets, but these systems do not facilitate search, understanding, or reuse without extensive, usually labor-intensive, curation to summarize or extract metadata \cite{Peer_2012}.
% Targeted repositories, on the other hand, generate large-scale, homogeneous data sources that facilitate meta- and re-analysis, but are limited to specific research questions, losing the full richness, context, and innovation of the source materials and field.

% Our approach is to focus on video and audio recordings in particular research domains, and build appropriate structured and unstructured resources around this functional core.
% By investigating of how researchers studying human development and education organize their own data, we have identified a unifying set of principles around which to structure contextual metadata.
% This structure facilitates the capture, standardization, discovery, and understanding of data to enable reuse of recordings at an unprecedented scale.
% We achieve this conformance through in-depth, investigative curation of previously collected datasets, along with familiar user interfaces allowing researchers to structure their own research data as they acquire it.
% In the future, we hope to capture additional annotation data researchers already produce while analyzing recordings to further improve our discovery capabilities.

\section{Data Architecture}

At the top level of organization, we separate individual data contributions into collections called \emph{volumes}.
Volumes contain a metadata (title, description, permissions, etc.) and supporting files (documents and other materials).
They are associated with persistent identifiers and can cite external resources by URI or Digital Object Identifier (DOI).

Most importantly, a volume involves two key structural elements: the \emph{session} and the \emph{record}.
Sessions provide the primary container for data files, an elaboration on traditional file folders with additional temporal metadata.
Records represent scalar variable-based data, traditionally contained in flat files or databases, and also provide an organizational structure among sessions.
The combination of these two elements allows data and metadata to be flexibly structured, related, and annotated.

\subsection{Sessions}

Sessions provide a container for raw materials and measurements that were made during a data collection \cite{Bakeman_2012}.
As recordings are often the core component of a data collection, a session may be defined by a single recording.
Or, a single session may comprise multiple videos, collected either serially in time representing different segments or in parallel from multiple cameras recording the same events.
These primary recordings may also be augmented with other forms of timeseries data such as motion capture, eye tracking, or physiological recordings, as well as with additional point-in-time measurements, observations, or surveys, all of which have specific and critical temporal relationships with the primary data.

When working with multiple files, researchers often combine them into a single representation.
For example, they may concatenate serial recordings or make a composite video by joining or overlaying frames.
Some may overlay or combine secondary data, by adding waveform or frequency visualizations below a video, or by overlaying cross-hairs representing object or eye movement.
Doing so allows for more convenient human analysis of the data in standard playback tools that support single data streams, but the result often suffers from degradation due to masking or downsampling.

A session captures relationships among data elements from a single data acquisition, representing files along with their absolute or relative temporal positioning.
We use a timeline metaphor: Each session defines an optional date and a time range, with absolute timestamps representing the exact time of data collection or relative positioning.
Files placed within a session can then be (optionally) positioned, much as in video or audio editing software.
Although some advanced recording equipment provides automatic timestamping or synchronization features, most researchers who collect parallel data streams use ``sync points'' such as an electronic trigger signal, light flash, or tone to align recordings.
These sync points can then be used to manually place recordings on the timeline.
In this way, the session represents recordings and related data streams in a way familiar to researchers, while keeping data in its most raw and flexible form.

\subsection{Records}

A record can represent many things typically conceived of as a row in a database.
Records may represent a single data sample, such as a participant, group, or classroom, a level of analysis, such as an experimental trial, task, condition, or location, or some other factor. 
% A record is thus defined by one of the above types along with any number of scalar values, which can be textual, numeric, or date.
In the case of a record representing an individual participant, records might include a subject identifier, gender, birth date, demographic information, or measurement data.
For other record types, these values may include independent variables, parameters, identifiers, or descriptions. 
We display records of a given type as a table, with the name of the variable as the column header and values populating the rows.

<<<<<<< HEAD:jcdl-15/paper-rog.tex
In addition to containing values, records also help to organize sessions.
Researchers organize data files in substantially different ways, often depending on the primary focus of analysis for a specific study or question.
=======
In addition to containing values, records are also used to organize sessions.
We have found that different researchers employ substantially different organizational structures for their files depending on their needs.
>>>>>>> c4ed7dc9b3368c9b3b1c2cf0bb24e87f7494f235:jcdl-15/paper.tex
Common directory structures include participants, session date, experimental condition, phase of analysis, and age group.
Sometimes single video files are split into segments according to scene or activity, and researchers consider this property the primary organizational principle.

Rather than trying to handle all of these strategies separately, Databrary leverages records to provide arbitrary groupings.
Records can be attached to sessions, or, equivalently, sessions can be grouped into records, in a many-to-many relationship.
For example, a study participant may appear in exactly one session, so the collected information about that participant, added to a record, is attached in a one-to-one relationship with a specific session.
Or, multiple participants may appear in multiple sessions, such as recordings of the same class of students over a span of time, and so the record may be attached to multiple sessions.
This similarly extends to any other groupings of sessions based on experimental procedures or other factors by attaching the appropriate record of the corresponding type.

Since researchers can create whichever record types apply to their data, they can also choose to group their data as they wish.
This allows users to import, browse, or export data in what they consider the most intuitive directory structure by indicating the appropriate nesting of record types
By standardizing across record types, we can also perform meta-searches across the library, for example for all videos containing participants within a particular age range.

\subsection{Annotations}

% Research using recordings typically proceeds with the addition of temporally-positioned annotations, either by direct observation or with the aid of simple visualization or analysis tools.
Specialized desktop software tools enable researchers to annotate events, measurements, utterances, or behaviors at any temporal scale, from individual samples or frames to the entire session.
Most tools focus on a single recording, and there is little interoperability among them.
Moreover, with rare exceptions in specific research domains \cite{MacWhinney2001}, there are few standards for annotations themselves, even across studies performed by the same researcher.

Databrary's architecture addresses some of these issues. The session container creates a timeline to which temporally-positioned annotations may be attached. 
Currently we support structured annotations added by data contributors and simple tag, key-word, and comment annotations contributed by the community.
Eventually we plan to support the range of annotations performed during analysis, including annotation imports from existing software, annotations from a future web-based coding tool, and through automated analysis via our API.

% First, records can be attached not only to entire sessions, but also to arbitrary temporal sub-sets of a session.
% If any record relates to a part, scene, or section of a session---for example a participant being present for only part of a recording---it can be applied within the timeline of a session to only that appropriate part.
% This allows an additional type of organizational structure researchers employ, wherein records define virtual sub-sessions that can be grouped and interacted with as units, without needing to cut up source videos.
% This can also theoretically be used to represent annotations at a finer temporal scale, for example individual trials or events.
% However, the appropriate interfaces and tools to input or import these annotations are still being developed.

% Second, we allow users to contribute simple keyword-based tagging and comment-based discussions on any temporal segment of a session as they watch the recordings.
% Such annotations can also be added through our API by automated video analysis according to features of interest identified in the data.
In addition to enriching datasets and enabling dialog between researchers, we expect annotations to provide an index for discovery capabilities when users look for particular features within data \cite{Lanagan_Smeaton_2012}.
We will also evaluate automated approaches for seeding tags based on existing metadata that are subsequently refined by the user community \cite{Yang_Lu_Giles_2011,Farooq_etal_2007,Giles2013}.

\subsection{Technical infrastructure}

%%dg - voting to put this and the repository section up higher in the paper, Approach, maybe?
Databrary is a cohesive web application, built in Scala on the Play Framework\footnote{http://playframework.com/} to enable a responsive user interface, a complete API, and high-performance streaming.
Recordings and other files are placed in content-based filesystem storage, and all structured data are stored in a PostgreSQL database, leveraging its geometric indexing capabilities for temporal data.
All uploaded recordings are automatically transcoded to a standard format to enable cross-platform HTML5-based streaming and downloading for off-line access, currently H.264/AAC in an MPEG-4 container for video.
This transcoding utilizes the high performance computing cluster at the host campus of New York University, using ffmpeg's libav bindings for both this and direct access to video frames and clips.
The user interface is built primarily on the Angular web framework\footnote{http://angularjs.org/}, and all data access is performed through an open JSON API.
Although hosted data are protected, all source code is released under a GPLv3 license on github\footnote{http://github.com/databrary}.

\section{Repository}

Databrary accepted contributions beginning in early 2014 and opened for general use in October 2014.
It currently hosts 5,700 video files totaling 1,600 hours of recordings along with 2,200 additional files.
These files comprise 2,400 sessions and are covered by 1,300 records (including 1,100 individual participants).
Data originates from 35 individual contributors across 25 different universities.

\subsection{Normalization}

Since we primarily deal with video and audio data, standardizing on an encoding format allows for convenient, consistent access without much loss of flexibility, while still allowing arbitrary resolutions, encoding qualities, and sample/frame-rates.
We additionally allow users, using certain portable formats (flat files, documents, images), to include data is not encompassed by our standard formats, but which may still be relevant for reuse.
Beyond that, we have introduced a number of metadata concepts where this tension becomes most apparent, namely the available record types, variables, and tag keywords.
We have adopted an approach that encourages adherence to existing standardized options but still allows new types to be created as the data require it.
Initially we have achieved this through manual review and off-line curation of contributions, dynamically extending the system as new types of data were discovered without burdening or constraining researchers.

However, since we expect the bulk of the data in the library to come from newly collected sources (due largely to privacy constraints on existing datasets), we have additionally taken a distinct, user-driven approach.
We found that while some researchers in our target domains have standardized data management practices, a large majority have no special data management or collection tools, often using a combination of hand-written paper, simple spreadsheets, existing video annotation tools, statistical analysis software, and local filesystem storage.
This suggested the opportunity to supply a cohesive data management platform where all collected raw data, including recordings, contextual information, and measurements, could be entered, stored, and exported for later off-line analysis.
Providing a targeted user interface to meet the existing and evolving needs of our contributed data allows us to control the means of data entry and thus achieve a greater amount of normalization in the data from the outset, through liberal use of auto-completion, suggestions, and drop-down options.
With the considerable benefit of reducing storage requirements within researcher's labs, and often adding convenience over existing practices, this has proven an attractive prospect for researchers.

\subsection{Discovery}

While increased normalization facilitates filtering and targeted searches for data, the general problems of browsing and discovery remain largely unsolved.
This problem is compounded for us because we expect that data collected in one research area, to be useful for entirely different purposes.
Thus, data arrives labeled with contextual information that may not fully describe the range of features present.
Ultimately, making determination of usefulness requires direct, human observation of the recordings, and so a large part the discovery solution will be simply to present researchers with short, representative samples from a variety of datasets.

Another important source for search targets would come from annotations supplied both by the original researchers during analysis as well as by the community.
In the former case, normalization remains a concern, as most annotation schemes researchers employ involve opaque, symbolic or numeric codes with little semantic content.
We are currently working to support import and export workflows for existing annotations tools that will be able to add meaningful definitions to these codes while researchers continue their current practices.
Beyond that, we also plan to build a new, more efficient and structured annotation web interface enabling more of the analysis workflow directly within our system.
In this way, Databrary will allow researchers to work with their recordings directly in raw form without having to transfer or transform them, while also capturing more meaningful and standardized contextual information from this process.

\section{Conclusions}

Starting from the perspective of a general-purpose repository, we have introduced a number of  restrictions to create one specialized for specific types of data.
By focusing on video and audio recordings we can normalize presentation and discovery of data across the site, and also introduce temporal structure to data and annotations driven by the nature of these recordings.
By limiting both file and metadata organization to a flat (but overlapping) set of sessions and records, we capture session-based research data within a cohesive and intuitive user interface.
Although we allow additional data files, for the time being we encourage users with other types of rich data, such as physiological recordings, to store and link to those resources externally.
Similarly, by targeting reuse of existing research data, rather than replication or meta-analysis, we limit the scope of accepted data to the original recordings and early analysis phases that generate the most value.

Having made significant progress building ingest and data entry services that have let us grow the library, we now must turn our efforts towards discovery and reuse, finding ways to better describe and elaborate on collected data.
To do this we look both to the original data contributors for ways that their existing analysis workflows can generate usable content descriptions, and to the community of researchers to label and comment on datasets as they explore.
Thus we seek to incorporate these existing annotation processes into Databrary, both to capture and define these annotations as generalizable search terms, and to make these workflows more convenient by obviating existing file transfer and transformation burdens.
Ultimately we hope to allow and capture instances of data reuse in this same way, so that content descriptions are aggregated from multiple annotation passes from different users and existing data can continuously increase in value for researchers.
%Achieving a sustainable balance will no doubt be challenging, but by using the above approaches and limiting our focus to particular research domains we have been able to achieve substantial standardization across the entire data library without overly burdening contributors.

We note that there is active discussion around the role of repositories and libraries in archiving and providing access to research data, where the roles and responsibilities between libraries and research departments is not entirely established \cite{Castelli_etal_2013, Nielson_Hjørland_2014, Macmillan_2014, Pinfield_etal_2014}.
Although resolving this complicated issue is not directly within our purview, we have laid out a functioning system for research data management that may bring data closer to established library standards that achieve these goals.
Databrary may be well-positioned to provide interoperability with library-based metadata schemas (such as export of data packages cross-walked to Dublin Core or METS schemas) and reach OAI-PMH compliance, so as to automatically incorporate the data that researchers add to Databrary into federated library searches with other domain specific data repositories.
By providing a refined API and assigning Digital Object Identifiers (DOIs) to volumes, we can also allow libraries and other information systems to tap into Databrary datasets.

Databrary's focus on making recordings from social and behavioral research available to and reusable by a community of scientists has provided perspectives we hope can inform the development of domain-specific research repositories in other domains. 
In turn, we look to leverage established knowledge in the digital library community in order to make Databrary an even more effective tool. 

\section*{Acknowledgments}

This work was supported by the NSF (BCS-1238599) and the NICHD (U01-HD-076595-01).
The authors gratefully acknowledge the NYU Libraries for their valuable advice and consultation.

\bibliographystyle{abbrv}
\bibliography{references}

\end{document}
