\documentclass{sig-alternate}

\begin{document}
\conferenceinfo{JCDL}{'15 Knoxville, Tennessee USA}

\title{Databrary: Enabling sharing and reuse of research video}

\numberofauthors{4}

\author{
\alignauthor
Dylan A. Simon\\
	\affaddr{Databrary Project}\\
	\affaddr{New York University}\\
	\affaddr{New York, NY, USA}\\
	\email{dylan@databrary.org}
\alignauthor
Andrew Gordon\\
	\affaddr{Databrary Project}\\
	\affaddr{New York University}\\
	\affaddr{New York, NY, USA}\\
	\email{drew@databrary.org}
\alignauthor
Lisa Steiger\\
	\affaddr{Databrary Project}\\
	\affaddr{New York University}\\
	\affaddr{New York, NY, USA}\\
	\email{lisa@databrary.org}
\and
Rick O. Gilmore\\
	\affaddr{Department of Psychology}\\
	\affaddr{Penn State}\\
	\affaddr{University Park, PA, USA}\\
	\email{rogilmore@psu.edu}
}

\maketitle

\begin{abstract}
Video and audio recordings serve as a primary data source in many fields, especially in the social and behavioral sciences.
Recordings present unique opportunities for reuse and reanalysis for novel scientific purposes. Though they also present a challenge in that they are often more identifiable than data provided in other formats.
Databrary is a web-based service for sharing and reusing the video data created by researchers in the developmental and learning sciences.
By investigating how researchers organize, analyze, and mine their own recordings, we have implemented a system that empowers researchers to capture, store, and share recordings in a standardized way.
This demo will provide a tour through the Databrary service, highlighting how it promotes storage, sharing, and data management for research data, controls user access to restricted data based on human research subjects, and facilitates browsing and discoverability of datasets.
\end{abstract}

\category{H.2.8}{Database Management}{Database Applications}[scientific databases, image databases]
\category{H.3.5}{Information Storage and Retrieval}{Online Information Systems}[data sharing, web-based services]

\terms{Design, Standardization}

\keywords{Data sharing, open science, video, psychology, developmental science}

\section{Introduction/Background}

Video and audio recordings serve as a primary data source for scientific research in psychology, linguistics, education, anthropology, and many other disciplines.
They are, in contrast to most timeseries data, human-consumable and largely self-documenting, and are often analyzed without the use of sophisticated tools, as human observers can directly glean much of the data's richness.
This allows recordings collected in one context to be used by others for different purposes with minimal explanation.

While video data has many benefits, it also presents a challenge in being highly identifiable.
Video often depicts participants’ faces, and their names are often spoken aloud. 
Because of this, sharing video data presents challenges related to the privacy of human subjects. 
In contrast to other forms of data which are non-identifiable or can be easily de-identified, video data cannot be easily de-identified without significantly diminishing its value and richness. 
Therefore, any solution for video sharing must provide appropriate policy and access frameworks to ensure appropriate protection of research participants.

Despite their inherent reusability, researchers rarely share recordings with others, due largely to entrenched practices and perceived privacy requirements.
This lack of sharing inhibits the reuse of recordings and impedes scientific discovery.
To address this problem, the Databrary project\footnote{https://databrary.org/} aims to build tools and policies that facilitate repurposing and reuse of recordings without placing undue burden on data contributors.
We have created a web-based data library to store and organize recordings and to capture contextual information necessary for reuse in standardized ways.

Our approach is to focus on video and audio recordings in a particular research domain, and build appropriate structured and unstructured resources around this functional core.
We designed the system by investigating how researchers studying human development and education organize their own data, deriving from those observations a unifying set of principles for organizing contextual metadata.
Thus it facilitates the capture, standardization, discovery, and understanding of data to enable reuse of recordings at an unprecedented scale.

\section{Technical details}

%we should probably update these stats.
Databrary began accepting contributions in early 2014 and opened for general use in October 2014.
It currently hosts 5,700 video files totaling 1,600 hours of recordings along with 2,200 additional files.
These files make up 2,400 sessions and are covered by 1,300 metadata records (including 1,100 individual participants).
Data originates from 35 individual contributors across 25 different universities.

Databrary is an open-source web application\footnote{http://github.com/databrary/databrary}, featuring a responsive user interface, a RESTful API, and high-performance video streaming.
Databrary stores at least two versions of each item of Databrary video content: a copy for access, and the received original file.
The access copy is generated by automatically transcoding the uploaded file to a standard format to enable cross-platform HTML5-based streaming and downloading for off-line access, currently H.264/AAC in an MPEG-4 container for video.

\section{Data Management for a Designated Community}

Focusing on a particular research domain allows us to develop a tool that is informed by how the intended community of users currently works, while enhancing that labor and adhering to the concerns and priorities for data that often has a high disclosure risk.
This has resulted in the creation of a set of data management features that empower researchers to actively manage their own projects---to upload data with accompanying metadata---as each study unfolds. 
We have designed and implemented a spreadsheet interface for entering, editing, and viewing session-level metadata (e.g., participants, conditions of study, tasks in the experiment, session access levels, study groups and conditions, etc.). 
Most researchers use desktop spreadsheets for precisely this purpose in their own labs, making the interface and functionality intuitive to users. 
We have also implemented a timeline for uploading, viewing, and tagging video assets related to sessions.
The timeline view is designed to look and operate like video editing software commonly used in many research labs.
It allows users to upload video files, position them to reflect the temporal order of each component of data collection, and annotate video sections with user-generated tags. 
These tags become additional metadata indices for search and discovery.

Providing a targeted user interface to meet the existing and evolving needs of our contributed data allows us to control the means of data entry and thus achieve a greater amount of normalization in the data from the outset, through liberal use of visualization, auto-completion, and suggestions.
In turn, this adds convenience and functionality for researchers over existing practices and alleviates storage burdens.

Because video data involving human subjects contains many points of identifiability, a core challenge to storing and sharing this type of research data is creating the access checks necessary to honor the original participant consents under which the data were collected. 
User registration for gaining access to Databrary also involves an authorization process by which site users are verified to be faculty, student, or researcher affiliated with an educational institution.
Users are also required to review and sign an access agreement stating that they will treat the data they are able to access through Databrary in a manner that complies with the ethical handling of research data involving human subjects\footnote{https://databrary.org/policies/agreement.pdf}.
Finally, contributors are offered release levels for files and groups of files that allow them to fine-tune who has access to their data. 

\section{Discovery and Scholarly Communication}

Though Databrary also strongly encourages sharing and reuse as a complement to data storage and management.
As such, we have put a priority on creating an interface that allows other researchers to find, filter, and download video files they might be able to use in their own research or classroom teaching. 
Currently basic text search over the entire repository is supported and soon we will begin to implement more refined search that allows faceting of search results based on tags, keywords, and session metadata such as participant age group, task, study condition, and location.
Additionally, we allow for the creation of tags that annotate videos and video segments, which will also serve as an index point for this more refined search functionality.  
Each dataset comes with a formatted citation, a persistent URI, and soon will also mint and provide a DOI for referencing in scholarly publications and presentations.
Based on input from site users we have also built in the ability to excerpt or highlight clips from videos in such a way that they might serve as an exemplar clip for the dataset or be used in the classroom as an education tool or at professional conferences.
As we continue to hear feedback from our intended user base, we not only see Databrary as a tool that facilitates the management of research data, but also as an opportunity to enhance the ability for academics to communicate about their research in all of its richness.

\section*{Acknowledgments}

This work was supported by the NSF (BCS-1238599) and the NICHD (U01-HD-076595-01).
The authors gratefully acknowledge the NYU Libraries for their valuable advice and consultation.

\bibliographystyle{abbrv}
\bibliography{references}

\end{document}
