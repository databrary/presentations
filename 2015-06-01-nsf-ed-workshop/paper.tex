\documentclass[letterpaper,man,natbib]{apa6}

\usepackage[english]{babel}
\usepackage[utf8x]{inputenc}
\usepackage{amsmath}
\usepackage{graphicx}
\usepackage[colorinlistoftodos]{todonotes}

\title{The Role of Video Data Sharing in Advancing Education Policy and Practice}
\shorttitle{Video Data Sharing in Education}
\author{Rick O. Gilmore}
\affiliation{The Databrary (databrary.org) Project \\ \& The Pennsylvania State University}

\abstract{Your abstract here.}

\begin{document}
\maketitle

\section{Introduction}

Video is a uniquely rich medium for capturing in real time the interactions that go on in classroom, laboratory, and informal learning settings. 

\section{The Promise of Video Data Sharing}

\subsection{Video is Uniquely Rich}

\subsection{Video Uses in Education}

\section{The Challenges of Video Data Sharing}

\subsection{Technical}

\subsubsection{Online Video Sharing}

\begin{itemize}
\item Size of files
\item Storage/streaming
\item Formats/transcoding
\item Preservation
\end{itemize}

\todo[inline, color=green!40]{Grab text from other papers?}

\subsubsection{Tools for Coding Video}

Human coders evaluate, apply text-based labels to videos using a wide-range of commercial and academic coding tools: Transana, StudioCode, V-code, MaxQDA, Noldus Observer, Mangold Interact, Datavyu.
No standard data format for coding tools.
Some offer web/cloud based storage (e.g., Transana, V-code), but otherwise coded data may not be easily shared within tools, much less between them.
With few exceptions, machine learning has not been applied to video data coding.

\subsection{Ethics and Privacy}

\subsection{Data Management}

\subsection{Changing Community Practices}

\section{Databrary.org}

\subsection{System Design}

\subsection{Policies for Safe, Secure, \& Sharing}

\subsection{Managing Data for Sharing}

\section{Conclusion}

\bibliography{example}

\end{document}