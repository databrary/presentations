A brash new kid has moved in down the street, and she has stolen the limelight from the old guard.
She's a data scientist, and she's teaching the old dogs of statistics some new tricks about how to detect, describe, and visualize patterns in complex sets of data.
Like successful newcomers in many fields, data scientists build on some well-established ideas and tools, but combine and apply them in new ways to new problems \cite{Press2013b}.
And, like many upstarts in recent memory, data scientists have ridden the wave created by the massive growth in the availability of information over the internet \cite{Borgmann2015}.
In this paper, I will focus on how the increased availability of and interest in analyzing large, complex data sets has influenced the study of human development.
I will begin by discussing the recent history of ``big data" and how, in principle, it has implications for developmental science.
Then, I will survey a set of case studies that illustrate how large, complex data sets have been and are being used currently.
To close, I will discuss a range of issues that the emergence of ``big data" approaches to developmental science pose for the future.

\subsection{Recent History}

Much of the public attention focused on big data concerns the influence of social media \cite{meyer_everything_2014}, impacts on economic development \cite{HBR2015}, or on data security breaches \cite{Vinton2015}.
he topic has also garnered the attention of government officials \cite{NSFBigData}, and private foundations \cite{ArnoldFdnEvidence, GatesFdn, MooreFdn} with interests in funding scientific research.
In March 2012, the U.S. Obama Administration announced an initiative to spend \$200 M dollars on a ``big data'' research and development (R\&D) initiative \cite{Gianchandani2012, Obama2012} with the expressed intent of applying data science techniques to large collections of digital data in order ``to solve some of the Nation's most pressing challenges..." \cite{Obama2012}.
The press release that accompanied the announcement \cite{Obama2012} highlighted some of the big data initiatives already or soon-to-be underway at the federal level, and these include the U.S. National Science Foundation (NSF), U.S. National Institutes of Health (NIH), the Departments of Energy and Defense and the U.S. Geological Survey (USGS).
The European Union (EU) has also taken public positions \cite{EU2013} about the importance of data to fostering social and economic well-being and has made it a priority to grant more open public access to data sets collected or managed by government entities \cite{EU2015}.
These potential benefits for open access to public data extend to governments in the developing world \cite{Malik2015}, and to state and local governments in the U.S.
For example, Indiana has embarked on an effort to reduce infant mortality using big data \cite{Ravindranath2014}.

Notwithstanding the promises of big data to improving public well-being, governments recognize that privacy and security concerns loom large \cite{Obama2014, Vinton2015} and must be addressed to fully realize the promise of big data science, especially in fields focused on human well-being.
For researchers, the promise of government interest in big data extends beyond potentially new or expanded funding sources \cite{NSFBigData}, such as the NIH's Big Data to Knowledge Initiative \cite{BD2K2015} or the now closed National Children's Study \cite{NCS2015}.
It means potentially new or more widely available or readily accessible data sources in open respositories \cite{databrary.org, dataverse.org, ICSPR, macwhinney_childes_2001} accompanied by more stringent requirements for open data sharing from funders \cite{NIMH2015, NSF2011, GatesFdn} and scientific publication outlets \cite{nosek_promoting_2015}.
At the same time, the collection, storage and open sharing of large data sets about pose significant data management challenges that may require changes in technology use \cite{ProjectJupyter, osf} and scientific practice, including closer collaboration between researchers and librarians and information specialists with expertise in data management, storage and long-term preservation \cite{gordon_researcher-library_2015, lynch_big_2008}.
We will revisit these topics later.

On the other hand, developmental scientists have long recognized the importance of collecting and sharing data sets that are big along at least one dimension -- the number of individuals, number of time points sampled, or the number of measures collected \cite{macwhinney_childes_2001}.
In turn, researchers in developmental science continue to pioneer advanced analytical techniques that unpack and explain developmental phenomena across multiple levels of analysis and across wide-ranging timescales \cite{bakeman_sequential_2011, CDS2014, QuantDev}.
Developmental scientists have also been at the vanguard of efforts to create ways to share research data that preserve privacy and uphold ethical research principles \cite{Adolph2012, AERA2011, AERAVideo2015, Asilomar2014, Databrary2015}.
Accordingly, the developmental research community may be especially well-positioned to turn big data into deep insight.

\subsection{Big data about what and for what purposes?}

Armed with some definitions about what constitutes big data, we can next ask what topics large datasets can describe and help inform.
The answer spans the range of phenomena covered by developmental science, from genes to geography and from microseconds to millenia.
In the case studies to follow, we will highlight studies that focus on biological evidence from genes, brain structure and activity patterns, hormonal variation, nutrition, growth, and health.
Biological evidence may derive from typically developing or atypically developing populations or both.
% Add citations to examples %
Often, behavioral evidence accompanies biological evidence, sometimes featuring standardized task assessments.
But, there are many examples of big datasets consisting solely of behavioral data.
These provide evidence about patterns of change and stability across the range of phenomena that interest developmental researchers: perception, action, cognition, language, emotion, sleep, personality, temperament, friendship, family, play, school achievement, substance use, sexuality, and sleep.
Still other large datasets focus on educational and employment aptitudes and outcomes and the social, economic, political, cultural, and historical contexts in which children and youth come of age.
% Again find examples and cite %

Large datasets can demonstrate either human diversity or human universals, depending on the topic and area of focus.
In most cases, however, big datasets in developmental science serve to answer core scientific questions that drive research in this area: What develops? 
Why does it develop that way? 
Does the environment change or the person or both, and how so?
High volume or velocity data may inform the estimation of trends within or between people across time \cite{rietveld_replicability_2014}.
Finally, many, perhaps most big datasets in developmental science have as a primary aim the understanding or treatment of disease or disorder.
In short, big data in developmental science focuses on answering the biggest, most important questions in the field.
In the next section, we turn to evaluating specific cases of research projects that have generated big datasets and data repositories that preserve, store, and share them with the scientific community.

This diversity in the targets of analysis reflects an implicit shared belief that developmental trajectories arise from a variety of influences organized at multiple levels of analysis operating across multiple time scales.

For example, patterns of brain activity change rapidly, while working memory spans and genotypes typically do not.
Further, brain activity and peripheral autonomic nervous system measurements like heart rate, heart period variability, and galvanic skin responses are related to central nervous system activity in causal, but indirect and complex ways. 
